\chapter{Thesis structure and text formatting}

This chapter explains some of the basics for formatting text in \LaTeX.

\section{Structuring the thesis}

To help \LaTeX\ understand the structure of your thesis, and so make it
possible for it to automatically generate a TOC you have to follow
some formatting rules.

\subsection{Chapter}

Each chapter is started by a:
\begin{verbatim}
\chapter{Name of chapter}
\end{verbatim}
So in each $.tex$ file you include, always start with a
\verb+\chapter{Whatever}+ line.

\subsection{Sections}
Sections, subsections and subsubsections are made like this:
\begin{verbatim}
\section{Name of section}
\subsection{Name of subsection}
\subsubsection{Name of subsubsection}
\end{verbatim}

\subsection{Numbering}
The numbering of the chapters, sections etc. should be left to \LaTeX\
and not you. You don't want to write stuff like:
\verb+\chapter{1. Introduction}+ this should all be left to
\LaTeX. The style of the numbering is set in the $thesis.tex$ file.

\section{Formatting text}

Here are the most basic ways of formatting text in \LaTeX:
\begin{itemize}
\item {\em Italic} is made like: \verb+{\it the text}+ or
\verb+{\em the text}+.
\item {\bf Bold} is made like: \verb+{\bf the text}+
\item \underline{Underline} is made like: \verb+\underline{the text}+
\end{itemize}

To set the size of some text you can choose between:
\begin{itemize}
\item {\normalsize normal size} \verb+{\normalsize normal size}+
\item {\large large} \verb+{\large large}+
\item {\Large larger} \verb+{\Large larger}+
\item {\LARGE larger still} \verb+{\LARGE larger still}+
\item {\huge huge} \verb+{\huge huge}+
\item {\Huge The hugest} \verb+{\Huge The hugest}+
\item {\small small} \verb+{\small small}+
\item {\footnotesize smaller than small} \verb+{\footnotesize smaller than small}+
\item {\scriptsize smaller still} \verb+{\scriptsize smaller still}+
\item {\tiny tiny} \verb+{\tiny tiny}+
\end{itemize}


\section{Lists}

\subsection{Lists}

You can make two kinds of lists in \LaTeX. Lists without or with
numbering. Unnumbered lists are made like:

\begin{verbatim}
\begin{itemize}
\item Red
\item Green
\item Blue 
\end{itemize}
\end{verbatim}

\begin{itemize}
\item Red
\item Green
\item Blue 
\end{itemize}

Numbered lists are made like:

\begin{verbatim}
\begin{enumerate}
\item Metal Gear Solid
\item System Shock
\item International Karate
\end{enumerate}
\end{verbatim}

\begin{enumerate}
\item Metal Gear Solid
\item System Shock
\item International Karate
\end{enumerate}

The text after \verb+\item+ can be as long as you want and multiple
lines are ok too.

\section{Text alignment and various breaks}

\subsection{Alignment}

Alignment of text etc. is done like:

\begin{verbatim}
\begin{flushleft}
This is left.
\end{flushleft}
\end{verbatim}

\begin{flushleft}
This is left.
\end{flushleft}

\begin{verbatim}
\begin{flushright}
This is right.
\end{flushright}
\end{verbatim}

\begin{flushright}
This is right.
\end{flushright}

\begin{verbatim}
\begin{center}
This is center.
\end{center}
\end{verbatim}

\begin{center}
This is center.
\end{center}

\subsection{Breaks and stuff}

If you want to force a new line you can either write: \verb+\\+ or
\verb+\newline+ 

To get a new page, do: \verb+\newpage+ 

For vertical spacing you have: \verb+\smallskip \medskip and \bigskip+ 

If you don't want the beginning of a text indented use: \verb+\noindent+
