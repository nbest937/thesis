\documentclass[draftthesis]{style/neiuthesis} 

\usepackage{natbib}
\usepackage[obeyDraft,prependcaption]{todonotes}

\ifpdf  
    \pdfinfo { /Title  (Best MA G\&ES NEIU Thesis)
               /Creator (TeX)
               /Producer (pdfTeX)
               /Author (Neil Best nbest@alum.mit.edu)
               /CreationDate (D:YYYYMMDDhhmmss)  %format D:YYYYMMDDhhmmss
               /ModDate (D:YYYYMMDDhhmm)
               /Subject (Land Use / Land Cover)
               /Keywords (agriculture, United States) }
    \pdfcatalog { /PageMode (/UseOutlines)
                  /OpenAction (fitbh)  }
\fi

\begin{document} \bibliographystyle{chicago}

\title{Synthesis of a complete land use / land cover data set for the
conterminous United States emphasizing accuracy in area and
distribution of agricultural activity} \author{Neil A. Best}
\degreeyear{December~2010}

\othermasters{Master of Arts}{M.A.}  \department{Geography \&
Environmental Studies}

\maketitle

\frontmatter

\begin{abstract} This paper presents an effort to produce a new land
cover data set for the conterminous United States that augments
available agricultural land use data with other uses and covers to
create a complete landscape characterization.  We start with the data
set described in Ramankutty, et al., 2008, which improves on the
spatial distributions of agricultural land indicated by the MODIS Land
Cover Type and Global Land Cover 2000 data products on which it is
based by incorporating agricultural census data as a ground truth
constraint.  However, it provides no information regarding areas which
were judged to have been misclassified.  We present a method for
reconciling the Ramankutty data with the MODIS Land Cover Type map for
2001 and aspects of the higher-resolution 2001 National Landcover
Database.  This result is subsequently merged with the data from
Monfreda, et al., 2008 in order to further disaggregate cropland into
commodity sub-classes.  We describe a prototype economic land use
change model driven by land conversion costs, crop yield expectations,
and climate change scenarios that requires this data for
initialization.  We examine trends in other data sets to assess the
accuracy of change in agricultural land area expressed in the MODIS
time series after 2001.  This effort points a way forward to the
eventual production of a global, annual time series of land cover/
land use maps featuring explicit disaggregation of croplands by
commodity to be used in economic modeling of global agricultural
production, trade, and consumption.
\end{abstract}

\chapter*{Acknowledgements}

This work was made possible through the support of my employer, the
Computation Institute, University of Chicago, and its director,
Dr. Ian Foster.


\tableofcontents \listoftables \listoffigures

%% Create a List of Abbreviations. The left column %% is 1 inch wide
and left-justified
\chapter{List of Abbreviations}

\begin{symbollist*}
\item[MODIS] Moderate-resolution Imaging Spectroradiometer
\item[MLCT] MODIS Land Cover Type \citep{MLCT}
\item[175Crops2000] Harvested Area and Yields of 175 crops (M3-Crops
Data), \citep{Monfreda2008}
\item[Aglands2000] \citep{Ramankutty2008}
\item [NLCD] National Land-Cover Database, 2001 \citep{Homer2004}
\item[arcmin] minute of arc, 1/60th of a degree
\item[arcsec] secondof arc, 1/60th of a minute, 1/3600th of a degree

\end{symbollist*}

%% Create a List of Symbols. The left column %% is 0.7 inch wide and
centered
\chapter{List of Symbols}

\begin{symbollist}[0.7in]
\item[$\tau$] Time taken to drink one cup of coffee.
\item[$\mu$g] Micrograms (of caffeine, generally).
\end{symbollist}

\mainmatter

There are some papers that say some important stuff
\citep{Ramankutty2008}


\chapter{Introduction}
\label{cha:introduction}

\section{Objective}
\label{sec:objective}

Recent years have seen a significant increase in the availability of
global land cover data sets inclding the UMD Global Land Cover
Classification product of 1998 \citep{Hansen2000}, Global Land Cover
2000 (GLC2000) \todo{GLC2000 citation}, 
MODIS Land Cover Type (MLCT) \todo{MLCT citation}.  MLCT stands out
among these due to its spatial resolution, nominally 500m, and its
distinction as a time series rather than a snapshot.  Economic models
of land use and land conversion require information that describes a
complete, albeit simplified, description of land cover and
land-intensive econommic activity in order to produce meaningful
statements and predictions about the evolution of land use patterns.
``Complete'' in this context means that all cover types or uses for a
given portion of land area are assigned a category in the model.
However while MLCT does satisfy this condition of completeness it
presents two new complications that we must first address.

The first is that MLCT presents an embarassment of riches in terms of
detail.  Regardless of its classification accuracy, which is discussed
below, the 15-arcsecond resolution, nominally 500m, is simply too much
information to be able to run the economic models in a reasonable
amount of time even on world-class high-performance computing
platforms.  A current standard resolution for global models of many
types and global data sets is 5-arcminutes, which is equivalent to a
400:1 pixel count reduction.  Other data sets featured in this
analysis use this resolution which is convenient for formulation.

The second requirement for the new complete land cover data set that
we wish to produce is that it provide greater information regarding
agricultural activity.  MLCT presents a single class for cropland but
we wish to further disaggregate the areas of agricultural production
according to a few major commodities in order to incorporate greater
detail of agronomic and commercial factors into the models.  As we
will see, \citet{Monfreda2008} provides a wealth of data in this
regard by harvested area and yield for 175 crops globally, but does
not provide a complete land cover description.  

\section{Tools}
\label{sec:tools}

A secondary objective of this paper is to demonstrate the capabilities
of a set of open-source geospatial, analytical, and publishing
software that includes \href{http://www.gdal.org/}{GDAL},
\href{http://grass.osgeo.org/}{GRASS} \citep{GRASS},
\href{http://www.r-project.org/}{R} \citep{R} , and
\href{http://www.latex-project.org/}{\LaTeX} \citep{Lamport1994} .
The last two members of this list are bridged by
\href{http://www.stat.uni-muenchen.de/~leisch/Sweave/}{Sweave}
\citep{Leisch2002} which allows embedding of analytical code written
in the R language within a \LaTeX document so that one step towards
producing a publication-quality PDF is running the analysis and
injecting its results directly into the content of the paper,
including tables, charts, and maps.  The underlying analysis code will
appear as an appendix.  This is a demonstration of reproducible
research as described in \citet{Gentleman2007}.

% -*- mode: noweb; noweb-default-code-mode: R-mode; -*-








\graphicspath{ {data/} }


\chapter{Data}
\label{cha:data}

This chapter presents some summary descriptions of the various data
sets that are relevant to this analysis and further discussion on how
they were manipulated.

The general approach with the classified land cover data sets (MLCT,
NLCD, CDL) is to reclassify their categories and aggregate the new
classification to the 5-arcmin grid.  The purpose of the
reclassification is to reduce the number of classes and have a uniform
set of classes across data sets.  The challenge in this is that
classification defintions are sometimes subtly different which makes
direct comparison across data sets difficult.  In this process we
convert classified maps whose pixels have discrete values to a stack
of maps, one map per class, whose pixels have real number values on
the interval $[0,1]$ and are constrained to sum to unity for each
pixel through the stack.  In the general case of the MLCT data product
the process converts two discrete, thematic variables and one
continuous variable, those being a primary covery type, a secondary
cover type, and classification confidence level respectively, into a
set of continuous variables representing fractional areas for the
cover types in the siplified classification system.  In other cases
the process is simplified by considering only a primary thematic layer
and performing the aggregation without a secondary cover type or
confidence level by which to relate them.

\missingfigure{Generate a summary table of data sets (raster/tabular, resolution, citation)}

\section{MODIS Land Cover Type (MLCT)}
\label{sec:mlct}

In preparation for this analysis we prepared the 2001 MLCT data by patching
together the tiles as delivered in the equal-area sinusoidal
projection, reprojecting that mosaic to geographic coordinates, and
extracting a subset for the conterminous United States (cUSA).  The
subset is defined as the set of 5-arcmin grid cells that intersect
with the cUSA polygon in the Global Administrative Areas (GADM) vector
data set, which includes the water bodies on the American side of the
international border across the Great Lakes, but does not extend to
oceanic waters beyond the coastal grid cells that intersect with the
land mass.  

The expectation is that the analytical approach developed here will be
applied globally in the future.

To illustrate the process of converting the MLCT data from its
original representation we are including maps of an area of
southeastern Michigan to show greater detail through each step of the
process.  We chose this region for its diversity of land covers and
uses, diveristy of agricultural commodities across its significant
cropland area, and its familiarity to our principal author, being his
brthplace.  In this section we will demonstrate the process of
converting the MLCT data from its native form, consisting of primary
cover type, classification confidence for the primary cover, and
secondary (alternate) cover type at 15-arcsec resolution, to a stack
of cover fractions at 5-arcmin resolution using the simplified
cover/use classification specified by the PEEL model. 

\subsection{Reclassification}
\label{sec:mlct-reclass}

The following table shows the mapping of the IGBP classes used in the
original MLCT data to the simplified classification designed for the
PEEL model.

\missingfigure{MLCT reclassification table}



% -*- mode: noweb; noweb-default-code-mode: R-mode; -*-


%\SweaveOpts{ eps=FALSE, pdf=FALSE, png=TRUE }

\graphicspath{ {analysis/} }

\chapter{Analysis}
\label{cha:analysis}


% latex table generated in R 2.11.1 by xtable 1.5-6 package
% Mon Dec  6 14:40:52 2010
\begin{table}[ht]
\begin{center}
\begin{tabular}{rrrrrrrrrr}
  \hline
 & crop & open & barren & forest & shrub & urban & water & wetland & mosaic \\ 
  \hline
As00 & 369.6 & 545.8 & 28.9 & 353.6 & 341.8 & 29.8 & 75.0 & 11.0 & 237.0 \\ 
  As05 & 379.2 & 516.3 & 32.9 & 344.5 & 359.1 & 27.3 & 73.4 & 26.1 & 232.8 \\ 
  ag & 446.5 & 557.1 &  &  &  &  &  &  &  \\ 
  agc & 450.1 & 558.3 & 25.3 & 450.3 & 368.0 & 35.1 & 74.5 & 30.9 &  \\ 
  nlcd & 310.8 & 429.6 & 24.5 & 513.2 & 420.1 & 102.8 & 96.5 & 95.0 &  \\ 
   \hline
\end{tabular}
\caption{Total Acreages by Map and Cover}
\label{tab:total}
\end{center}
\end{table}
The MLCT indicates 379.2Ma (153.4Mha) of
cropland in the cUSA in 2001. Assuming that 50\% of the
cropland/natural vegetation mosaic is additional cropland area gives
an additional 116.4Ma (47.1Mha)
of agricultural land. This gives a total area of 
495.6Ma (200.6Mha)
of total area directly associated with
agricultural activity according to the IGBP classification used in the
MLCT.

Aglands2000 indicates roughly 446.5Ma (180.7Mha) 
of cropland.

Pasture indicated by Aglands2000 appears to be a broader
classification than that of the NLCD's pasture class because much of
the grazing land east of the Mississippi river counted in the
Aglands2000 pasture map is absent in the NLCD pasture class.

Due to its greater resolution (30m) the NLCD is better suited at
discerning developed areas in rural landscapes ranging from rural
roads to farmsteads to small communities that do not show up in the
MLCT data. There is a total area of roughly 74 Ma (30 Mha) of
development remaining after subtracting the MLCT urban class from all
developed classes in the NLCD where the NLCD shows greater development
after they have both been aggregated to the 5-arcmin grid. Applying
this area as an offset to the cropland area in Aglands2000 brings us
closer to the expected acreage under cultivation in 2001, although
this assumes that all of that development intersects with MLCT
cropland area.


% latex table generated in R 2.11.1 by xtable 1.5-6 package
% Mon Dec  6 14:44:54 2010
\begin{table}[ht]
\begin{center}
\begin{tabular}{rrrrr}
  \hline
 & rmse\_frac & bias\_frac & rmse\_acres & bias\_acres \\ 
  \hline
barren & 0.05 & 2.68E-04 & 902 & 7 \\ 
  crop & 0.15 & 6.94E-02 & 2515 & 1142 \\ 
  forest & 0.20 & -2.82E-02 & 3249 & -516 \\ 
  open & 0.32 & 6.13E-02 & 5418 & 1056 \\ 
  shrub & 0.33 & -2.65E-02 & 5572 & -427 \\ 
  urban & 0.06 & -3.34E-02 & 973 & -555 \\ 
  water & 0.04 & -1.11E-02 & 580 & -181 \\ 
  wetland & 0.10 & -3.16E-02 & 1608 & -526 \\ 
   \hline
\end{tabular}
\caption{Errors and Biases of Aglands Complete relative to NLCD}
\label{tab:ebagc}
\end{center}
\end{table}% latex table generated in R 2.11.1 by xtable 1.5-6 package
% Mon Dec  6 14:44:54 2010
\begin{table}[ht]
\begin{center}
\begin{tabular}{rrrrr}
  \hline
 & rmse\_frac & bias\_frac & rmse\_acres & bias\_acres \\ 
  \hline
barren & 0.07 & 1.95E-03 & 1140 & 36 \\ 
  crop & 0.19 & 3.18E-02 & 2945 & 482 \\ 
  forest & 0.20 & -7.77E-02 & 3380 & -1309 \\ 
  open & 0.33 & 5.84E-02 & 5428 & 953 \\ 
  shrub & 0.29 & -4.34E-02 & 4783 & -642 \\ 
  urban & 0.06 & -3.62E-02 & 988 & -598 \\ 
  water & 0.03 & -1.09E-02 & 504 & -177 \\ 
  wetland & 0.10 & -4.15E-02 & 1759 & -689 \\ 
   \hline
\end{tabular}
\caption{Errors and Biases of MLCT, $A_s = 0.0$ relative to NLCD}
\label{tab:ebmlct00}
\end{center}
\end{table}% latex table generated in R 2.11.1 by xtable 1.5-6 package
% Mon Dec  6 14:44:54 2010
\begin{table}[ht]
\begin{center}
\begin{tabular}{rrrrr}
  \hline
 & rmse\_frac & bias\_frac & rmse\_acres & bias\_acres \\ 
  \hline
barren & 0.06 & 3.89E-03 & 1092 & 69 \\ 
  crop & 0.17 & 3.63E-02 & 2690 & 560 \\ 
  forest & 0.20 & -8.21E-02 & 3362 & -1383 \\ 
  open & 0.30 & 4.35E-02 & 4891 & 711 \\ 
  shrub & 0.27 & -3.43E-02 & 4410 & -500 \\ 
  urban & 0.06 & -3.74E-02 & 1027 & -619 \\ 
  water & 0.03 & -1.17E-02 & 529 & -190 \\ 
  wetland & 0.10 & -3.42E-02 & 1595 & -565 \\ 
   \hline
\end{tabular}
\caption{Errors and Biases of MLCT, $A_s = 0.5$ relative to NLCD}
\label{tab:ebmlct05}
\end{center}
\end{table}

\begin{figure} 
\begin{center} 
\includegraphics{fig_agc}
\end{center} 
\caption{Aglands Complete cover maps} 
\label{fig:agc} 
\end{figure} 

\begin{figure} 
\begin{center} 
\includegraphics{fig_nlcd}
\end{center} 
\caption{NLCD cover maps} 
\label{fig:nlcd} 
\end{figure} 

\begin{figure} 
\begin{center} 
\includegraphics{fig_diff}
\end{center} 
\caption{Difference maps, Aglands Complete minus NLCD} 
\label{fig:diff} 
\end{figure} 

\begin{figure} 
\begin{center} 
\includegraphics{fig_cordiff}
\end{center} 
\caption{Correlations across cover type in difference maps} 
\label{fig:cordiff} 
\end{figure} 

The elements of the matrix have been reordered according to the
clustering forumla given in \citet[sec. 6.2.3]{Sarkar2008} in order to
achieve a degree of visual clustering among the correlation vectors.

%%% Local Variables: 
%%% mode: latex
%%% TeX-master: "thesis"
%%% End: 


\chapter{Conclusions}

We conclude that graduate students like coffee.

\appendix*

%\include{appendix}

\backmatter

\bibliography{thesis}

\end{document}