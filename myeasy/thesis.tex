\documentclass[pdftex,letterpaper,11pt]{report}
%\usepackage[letter]{geometry}

%packages added to original template
\usepackage{natbib}

% packages originally included in template
%\usepackage[T1]{fontenc}
%\usepackage{makeidx}
%\usepackage{makeindex}
%\usepackage{alltt}
%\usepackage{tabularx}
% \usepackage{moreverb}
% \usepackage{url}
\usepackage[pstex]{graphicx}
% \usepackage{pslatex}
% \usepackage{mathptmx}
% \usepackage{amsmath}
% \usepackage{pstricks}

% just to kick in revert/reload in acroread (on unix)
\usepackage{hyperref}
\begin{Form}\end{Form}

% here we define a bold-italic command
\newcommand{\bolit}[1] {\item[\textbf{#1}] \mbox{} \\}

% here we define a new \figcaption{} command
\makeatletter
\newcommand\figcaption{\def\@captype{figure}\caption}
\makeatother

\title{Synthesis of a complete land use / land cover data set for the conterminous United States emphasizing accuracy in area and distribution of agricultural activity}
% \author{\href{mailto:nbest@alum.mit.edu}{Neil A. Best}}
\author{Neil A. Best}

\begin{document}
\bibliographystyle{chicago}

% the command below generates the title page (first page)
\maketitle

\begin{abstract}
This paper presents an effort to produce a new land cover data set for the conterminous United States that augments available agricultural land use data with other uses and covers to create a complete landscape characterization.  We start with the data set described in Ramankutty, et al., 2008, which improves on the spatial distributions of agricultural land indicated by the MODIS Land Cover Type and Global Land Cover 2000 data products on which it is based by incorporating agricultural census data as a ground truth constraint.  However, it provides no information regarding areas which were judged to have been misclassified.  We present a method for reconciling the Ramankutty data with the MODIS Land Cover Type map for 2001 and aspects of the higher-resolution 2001 National Landcover Database.  This result is subsequently merged with the data from Monfreda, et al., 2008 in order to further disaggregate cropland into commodity sub-classes.  We describe a prototype economic land use change model driven by land conversion costs, crop yield expectations, and climate change scenarios that requires this data for initialization.  We examine trends in other data sets to assess the accuracy of change in agricultural land area expressed in the MODIS time series after 2001.  This effort points a way forward to the eventual production of a global, annual time series of land cover/ land use maps featuring explicit disaggregation of croplands by commodity to be used in economic modeling of global agricultural production, trade, and consumption.
\end{abstract}

% the pages hereafter will be numbered I,II,III etc.
\pagenumbering{roman}
\small
% the part below includes an acknowledgments chapter that will not be numbered
\setcounter{secnumdepth}{-1}
\chapter{Acknowledgments}

This thesis could not have been made without the help from the following pople:

\noindent
Insert rant about tutor here ;)

\noindent
Another rant about your family.

\noindent
Some ppl you actually got help and emails from.

\setcounter{secnumdepth}{3}
% here we get a TOC, a table of contents
\tableofcontents
% here we get a list of figures/images
\listoffigures
% and here tables
\listoftables
\normalsize
\newpage
% now we're back to 1,2,3,4...
\pagenumbering{arabic}

% we also have an introduction chapter that will not be numbered
\setcounter{secnumdepth}{-1}

\chapter{Introduction}
\label{cha:introduction}

\section{Objective}
\label{sec:objective}

Recent years have seen a significant increase in the availability of
global land cover data sets inclding the UMD Global Land Cover
Classification product of 1998 \citep{Hansen2000}, Global Land Cover
2000 (GLC2000) \todo{GLC2000 citation}, 
MODIS Land Cover Type (MLCT) \todo{MLCT citation}.  MLCT stands out
among these due to its spatial resolution, nominally 500m, and its
distinction as a time series rather than a snapshot.  Economic models
of land use and land conversion require information that describes a
complete, albeit simplified, description of land cover and
land-intensive econommic activity in order to produce meaningful
statements and predictions about the evolution of land use patterns.
``Complete'' in this context means that all cover types or uses for a
given portion of land area are assigned a category in the model.
However while MLCT does satisfy this condition of completeness it
presents two new complications that we must first address.

The first is that MLCT presents an embarassment of riches in terms of
detail.  Regardless of its classification accuracy, which is discussed
below, the 15-arcsecond resolution, nominally 500m, is simply too much
information to be able to run the economic models in a reasonable
amount of time even on world-class high-performance computing
platforms.  A current standard resolution for global models of many
types and global data sets is 5-arcminutes, which is equivalent to a
400:1 pixel count reduction.  Other data sets featured in this
analysis use this resolution which is convenient for formulation.

The second requirement for the new complete land cover data set that
we wish to produce is that it provide greater information regarding
agricultural activity.  MLCT presents a single class for cropland but
we wish to further disaggregate the areas of agricultural production
according to a few major commodities in order to incorporate greater
detail of agronomic and commercial factors into the models.  As we
will see, \citet{Monfreda2008} provides a wealth of data in this
regard by harvested area and yield for 175 crops globally, but does
not provide a complete land cover description.  

\section{Tools}
\label{sec:tools}

A secondary objective of this paper is to demonstrate the capabilities
of a set of open-source geospatial, analytical, and publishing
software that includes \href{http://www.gdal.org/}{GDAL},
\href{http://grass.osgeo.org/}{GRASS} \citep{GRASS},
\href{http://www.r-project.org/}{R} \citep{R} , and
\href{http://www.latex-project.org/}{\LaTeX} \citep{Lamport1994} .
The last two members of this list are bridged by
\href{http://www.stat.uni-muenchen.de/~leisch/Sweave/}{Sweave}
\citep{Leisch2002} which allows embedding of analytical code written
in the R language within a \LaTeX document so that one step towards
producing a publication-quality PDF is running the analysis and
injecting its results directly into the content of the paper,
including tables, charts, and maps.  The underlying analysis code will
appear as an appendix.  This is a demonstration of reproducible
research as described in \citet{Gentleman2007}.

\setcounter{secnumdepth}{3}
% insert chapters here!
\chapter{Thesis structure and text formatting}

This chapter explains some of the basics for formatting text in \LaTeX.

\section{Structuring the thesis}

To help \LaTeX\ understand the structure of your thesis, and so make it
possible for it to automatically generate a TOC you have to follow
some formatting rules.

\subsection{Chapter}

Each chapter is started by a:
\begin{verbatim}
\chapter{Name of chapter}
\end{verbatim}
So in each $.tex$ file you include, always start with a
\verb+\chapter{Whatever}+ line.

\subsection{Sections}
Sections, subsections and subsubsections are made like this:
\begin{verbatim}
\section{Name of section}
\subsection{Name of subsection}
\subsubsection{Name of subsubsection}
\end{verbatim}

\subsection{Numbering}
The numbering of the chapters, sections etc. should be left to \LaTeX\
and not you. You don't want to write stuff like:
\verb+\chapter{1. Introduction}+ this should all be left to
\LaTeX. The style of the numbering is set in the $thesis.tex$ file.

\section{Formatting text}

Here are the most basic ways of formatting text in \LaTeX:
\begin{itemize}
\item {\em Italic} is made like: \verb+{\it the text}+ or
\verb+{\em the text}+.
\item {\bf Bold} is made like: \verb+{\bf the text}+
\item \underline{Underline} is made like: \verb+\underline{the text}+
\end{itemize}

To set the size of some text you can choose between:
\begin{itemize}
\item {\normalsize normal size} \verb+{\normalsize normal size}+
\item {\large large} \verb+{\large large}+
\item {\Large larger} \verb+{\Large larger}+
\item {\LARGE larger still} \verb+{\LARGE larger still}+
\item {\huge huge} \verb+{\huge huge}+
\item {\Huge The hugest} \verb+{\Huge The hugest}+
\item {\small small} \verb+{\small small}+
\item {\footnotesize smaller than small} \verb+{\footnotesize smaller than small}+
\item {\scriptsize smaller still} \verb+{\scriptsize smaller still}+
\item {\tiny tiny} \verb+{\tiny tiny}+
\end{itemize}


\section{Lists}

\subsection{Lists}

You can make two kinds of lists in \LaTeX. Lists without or with
numbering. Unnumbered lists are made like:

\begin{verbatim}
\begin{itemize}
\item Red
\item Green
\item Blue 
\end{itemize}
\end{verbatim}

\begin{itemize}
\item Red
\item Green
\item Blue 
\end{itemize}

Numbered lists are made like:

\begin{verbatim}
\begin{enumerate}
\item Metal Gear Solid
\item System Shock
\item International Karate
\end{enumerate}
\end{verbatim}

\begin{enumerate}
\item Metal Gear Solid
\item System Shock
\item International Karate
\end{enumerate}

The text after \verb+\item+ can be as long as you want and multiple
lines are ok too.

\section{Text alignment and various breaks}

\subsection{Alignment}

Alignment of text etc. is done like:

\begin{verbatim}
\begin{flushleft}
This is left.
\end{flushleft}
\end{verbatim}

\begin{flushleft}
This is left.
\end{flushleft}

\begin{verbatim}
\begin{flushright}
This is right.
\end{flushright}
\end{verbatim}

\begin{flushright}
This is right.
\end{flushright}

\begin{verbatim}
\begin{center}
This is center.
\end{center}
\end{verbatim}

\begin{center}
This is center.
\end{center}

\subsection{Breaks and stuff}

If you want to force a new line you can either write: \verb+\\+ or
\verb+\newline+ 

To get a new page, do: \verb+\newpage+ 

For vertical spacing you have: \verb+\smallskip \medskip and \bigskip+ 

If you don't want the beginning of a text indented use: \verb+\noindent+

\chapter{Images and tables}

\section{Images}

To include an image in the text is fairly easy.

\begin{verbatim}
\begin{center}
\includegraphics{images/sinfest.png}
\figcaption{Cute, cute girl from www.sinfest.net.}
\end{center}
\end{verbatim}

\begin{center}
\includegraphics{images/sinfest.png}
\figcaption{Cute, cute girl from www.sinfest.net.}
\end{center}

Is is possible to scale the image by setting the width of it like
this:
\begin{verbatim}
\includegraphics[width=\textwidth]{images/sinfest.png}
\end{verbatim}
and several other ways. In the example above the image will keep its
constraints, and be scaled to be as wide as the documents text width.

\section{Tables}

Tables gets a bit more tricky in \LaTeX. What you do is decide the
number of columns and then use a seperator character, $\&$, to delimit
the different table cells. Example:
\begin{verbatim}
\begin{table}[tbph]
\begin{center}
\begin{tabular}{|c|c|c|}
\hline
{\bf Name}&{\bf Surname}&{\bf Age}\\
\hline
Bobba&Fett&42\\
\hline
Yoda&&900\\
\hline
\end{tabular}
\caption{Just these guys you know?}
\end{center}
\end{table}
\end{verbatim}

\begin{table}[tbph]
\begin{center}
\begin{tabular}{|c|c|c|}
\hline
{\bf Name}&{\bf Surname}&{\bf Age}\\
\hline
Bobba&Fett&42\\
\hline
Yoda&&900\\
\hline
\end{tabular}
\caption{Just these guys you know?}
\end{center}
\end{table}


\chapter{Citations and cross-references}

\section{Citations}
``I guess this text isen't actually in the following book, but who
cares anyway'' \citep{proakis96}.

\section{Cross references \label{crossreferences}}

In Section \ref{crossreferences} we use, uhm...  cross references.
% and finally a conclusion chapter
\chapter{Conclusion}

We need more funding!

\appendix
\chapter{Glossary}

\begin{description}

\bolit{Adaptive Differential Pulse Coded Modulation (ADPCM)}
A speech compression algorithm that adaptively filters the difference between
two successive PCM samples. This technique typically gives a data rate
of about 32 Kbps.

\bolit{adaptive filter}
A filter that can adapt its coefficients to model a
system.

\bolit{aliasing}
The effect on a signal when it has been sampled at
less than twice its highest frequency.

\end{description}

% the citations we have made in the text are listed here
\bibliography{books}
\addcontentsline{toc}{chapter}{Bibliography}

\end{document}
