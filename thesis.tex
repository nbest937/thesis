\documentclass[draftthesis]{neiuthesis} 

\usepackage{amsmath}
\usepackage{natbib}
\usepackage[prependcaption]{todonotes} %[obeyDraft]
\usepackage{Sweave}
% changes to Sweave.sty defaults
% from http://faculty.agecon.vt.edu/moeltner/AAEC5126/Sweave/docs/Arnholt_tutorial.pdf
\DefineVerbatimEnvironment{Sinput}{Verbatim}{fontsize=\small,fontshape=n}
\DefineVerbatimEnvironment{Soutput}{Verbatim}{fontsize=\small,fontshape=n}
\DefineVerbatimEnvironment{Scode}{Verbatim}{fontsize=\small,fontshape=n}
\usepackage{graphicx}
\usepackage{listings} 
\usepackage{courier} 
\lstset{breaklines=true,basicstyle=\ttfamily,language=R}
\usepackage{pdfpages}
\usepackage{pbox}
\usepackage{verbatim}
\usepackage{rotating}
\usepackage[normalem]{ulem}
\usepackage{hyperref}
\hypersetup{
  pdftitle={NEIU G\&ES MA Thesis},%
  pdfauthor={Neil Best <nbest@alum.mit.edu>},%
  pdfsubject={land use / land cover},%
  pdfkeywords={agriculture}{United States},%
  pdffitwindow=true,%
  pdfstartview={FitH}
}
\usepackage[all]{hypcap}
\usepackage{multirow}

\begin{document} 
\bibliographystyle{chicago}

\title{Synthesis of a complete land use\slash land cover data set for the
  conterminous United States emphasizing accuracy in area and
  distribution of agricultural activity}
 

\othermasters{Master of Arts}{M.A.}  
\department{Geography \& Environmental Studies}

\author{Neil A. Best}
\degreeyear{August~2011}



\maketitle

%\includepdf[pages={2,1}]{neiuForms.pdf}
%!pdfTeX error: pdflatex (file ./neiuForms.pdf): PDF inclusion: invalid font in reference type <dictionary>

\frontmatter

\begin{abstract} This paper presents an effort to produce a new land
  cover data set for the conterminous United States of America (cUSA)
  that augments available agricultural land use data with other uses
  and natural covers to create a complete landscape characterization.
  Using the Agland2000 data set as a benchmark we formulate a
  hybridization of the MODIS Land Cover Type (MLCT) for 2001 and the
  2001 National Land Cover Database (NLCD) that is particularly
  tailored to serve as an initialization data set for long-term
  economic land use change models.  In order to strike a balance
  between spatial precision and local diversity of use and cover the
  new data set has lower resolution than the MLCT ($5'$ vs. $15''$)
  but represents land use/land cover (LULC) components as sub-pixel
  fractions rather than discrete categories.  After aggregating to the
  $5'$ grid we present a method for decomposing the natural
  vegetation/cropland mosaic class found in MLCT into constituent
  classes as a function of the local landscape.  We then quantify its
  contribution to aggregate acreages by class, particularly cropland.
  We compensate for the absence of certain fine-grained details from
  MLCT, such as rural transportation networks, small settlements,
  linear water features, and wetlands, mainly due to sensor
  resolution, by incorporating corresponding components of the NLCD,
  after similar reclassification and aggregation, as a set of offsets
  to the MLCT-derived fractions.  The 175Crops2000 data set, valuable
  for its basis in per-crop agricultural production statistics, is
  used as a guide to further decompose the cropland areas into a set
  of crop-specific sub-categories designed to facilitate the economic
  modeling goals of the simulations that will be initialized by this
  data product.  The final classification scheme is conceptually
  equivalent to a stack of spectral bands with the additional property
  that the components of each pixel sum to unity.  Its classification
  scheme is a mixture of a simplified version of the IGBP schema used
  in MLCT and a disaggregation of the monolithic cropland class that
  differentiates among the world's major commodity crops.  At each
  step of refinement we show that overall spatial distribution of
  cropland across the study area improves relative to the Aglands2000
  data set.  We close with a discussion of how this method might be
  applied globally and to successive years in the MLCT time series.
\end{abstract}


\chapter*{Acknowledgments}

\noindent This thesis is dedicated to my son, Leo.  Son, I began
working on this degree before you were born and my commitment to
completing it was sustained by my desire to demonstrate to you that in
life we finish what we have started.

\vspace{12pt}
\noindent I could not have completed this paper over the past year
and, by extension, my degree over more years than I care to mention
without the support of my loving wife, Laura.

\vspace{12pt}
\noindent I want to thank Dr. Nicholas Kouchoukos of Lanworth,
Inc. for throwing me in the deep end of applying the open-source
geospatial software tool chain to spatial analysis of agriculture.

\vspace{12pt}
\noindent This work was made possible through the support of my
employer, the Computation Institute at the University of Chicago, and
its director, Dr. Ian Foster under the Community Integrated Model of
Economic and Resource Trajectories for Humankind project (CIM-EARTH,
\url{http://www.cim-earth.org/}).

\vspace{12pt}
\noindent My thesis committee was comprised of Dr. Monika Mihir
(chair), Dr. Erick Howenstine (department head), both of the
Department of Geography \& Environmental Studies at Northeastern
Illinois University, and Dr. Joshua Elliott from the Computation
Institute.  I deeply appreciate their guidance and support through all
stages of this project.


\tableofcontents 
\listoftables
\listoffigures

%% Create a List of Abbreviations. The left column %% is 1 inch wide and left-justified
\chapter{List of Abbreviations}

\begin{symbollist*}
\item[175Crops2000] Harvested Area and Yields of 175 crops (M3-Crops
  Data) \citep{Monfreda2008}
\item[Agland2000] Agricultural Lands in the Year 2000 (M3-Cropland and
  M3-Pasture Data) \citep{Ramankutty2008}
\item[AVHRR] Advanced Very High Resolution Radiometer
\item[CIM-EARTH] Community Integrated Model of Economic and Resource
  Trajectories for Humankind
\item[cUSA] conterminous (contiguous) Unites States of America, the ``lower 48''
\item[GADM] Global Administrative Areas, \url{http://www.gadm.org/}
\item[GIAM] Global Irrigated Areas Map
\item[GIS] Geographic Information Systems
\item[GMRCA] Global Map of Rainfed Crop Areas
\item[GLC2000] Global Land Cover 2000 \citep{EC2003}
\item[GRASS] Geographic Resources Analysis Support System, \url{http://grass.osgeo.org/}
\item[IGBP] International Geosphere-Biosphere Programme
\item[LULC] land use / land cover
\item[MODIS] Moderate Resolution Imaging Spectroradiometer
\item[MLCT] MODIS Land Cover Type \citep{MLCT}
\item[NLCD] National Land-Cover Database, 2001 \citep{Homer2004}
\item[PEEL] Partial Equilibrium Economic Land use model
\item[PLSS] Public Land Survey System
\item[RMSE] root of the mean squared error
\item[SPAM] Spatial Production Allocation Model
\item[SPOT] Syst\`eme pour l'Observation de la Terre

\end{symbollist*}

%% Create a List of Symbols. The left column 
%% is 0.7 inch wide and centered
\chapter{List of Symbols}

\begin{symbollist}[0.7in]
\item[$A_{min}$] Minimum sub-pixel fraction possible for primary cover
  given in MLCT base data
\item[$A_s$] Sub-pixel fraction of secondary cover type, function of
  classification confidence level and $A_{min}$
\item[$A_p$] Sub-pixel fraction of primary cover type, function of
  classification confidence level and $A_{min}$
\item[$\hat\theta$] Predicted sub-pixel fraction
\item[$\theta$] Observed sub-pixel fraction
\item[$'$] minute of arc, 1/60th of a degree
\item[$''$] second of arc, 1/60th of a minute, 1/3600th of a degree
\item[{[0,1]}] interval from 0 to 1, inclusive of 0 and 1; $0 \leq x \leq 1$ 
\item[[0,1)] interval from 0 to 1, inclusive of 0 but but not 1, $0 \leq x < 1$ 
%\item[$ \left[ 0,1 \right] $] interval from 0 to 1, inclusive of 0 and 1; $0 \leq x \leq 1$ 
%\item[$ \left[ 0,1 \right) $] interval from 0 to 1, inclusive of 0 but but not 1, $0 \leq x < 1$ 
\end{symbollist}

\todo{fix interval/set-builder notation in symbol list}

\mainmatter

\todototoc
\listoftodos


\chapter{Introduction}
\label{cha:introduction}

\section{Objective}
\label{sec:objective}

Recent years have seen a significant increase in the availability of
global land cover data sets inclding the UMD Global Land Cover
Classification product of 1998 \citep{Hansen2000}, Global Land Cover
2000 (GLC2000) \todo{GLC2000 citation}, 
MODIS Land Cover Type (MLCT) \todo{MLCT citation}.  MLCT stands out
among these due to its spatial resolution, nominally 500m, and its
distinction as a time series rather than a snapshot.  Economic models
of land use and land conversion require information that describes a
complete, albeit simplified, description of land cover and
land-intensive econommic activity in order to produce meaningful
statements and predictions about the evolution of land use patterns.
``Complete'' in this context means that all cover types or uses for a
given portion of land area are assigned a category in the model.
However while MLCT does satisfy this condition of completeness it
presents two new complications that we must first address.

The first is that MLCT presents an embarassment of riches in terms of
detail.  Regardless of its classification accuracy, which is discussed
below, the 15-arcsecond resolution, nominally 500m, is simply too much
information to be able to run the economic models in a reasonable
amount of time even on world-class high-performance computing
platforms.  A current standard resolution for global models of many
types and global data sets is 5-arcminutes, which is equivalent to a
400:1 pixel count reduction.  Other data sets featured in this
analysis use this resolution which is convenient for formulation.

The second requirement for the new complete land cover data set that
we wish to produce is that it provide greater information regarding
agricultural activity.  MLCT presents a single class for cropland but
we wish to further disaggregate the areas of agricultural production
according to a few major commodities in order to incorporate greater
detail of agronomic and commercial factors into the models.  As we
will see, \citet{Monfreda2008} provides a wealth of data in this
regard by harvested area and yield for 175 crops globally, but does
not provide a complete land cover description.  

\section{Tools}
\label{sec:tools}

A secondary objective of this paper is to demonstrate the capabilities
of a set of open-source geospatial, analytical, and publishing
software that includes \href{http://www.gdal.org/}{GDAL},
\href{http://grass.osgeo.org/}{GRASS} \citep{GRASS},
\href{http://www.r-project.org/}{R} \citep{R} , and
\href{http://www.latex-project.org/}{\LaTeX} \citep{Lamport1994} .
The last two members of this list are bridged by
\href{http://www.stat.uni-muenchen.de/~leisch/Sweave/}{Sweave}
\citep{Leisch2002} which allows embedding of analytical code written
in the R language within a \LaTeX document so that one step towards
producing a publication-quality PDF is running the analysis and
injecting its results directly into the content of the paper,
including tables, charts, and maps.  The underlying analysis code will
appear as an appendix.  This is a demonstration of reproducible
research as described in \citet{Gentleman2007}.

s% -*- mode: noweb; noweb-default-code-mode: R-mode; -*-








\graphicspath{ {datasets/} }


\chapter{Data Sets}
\label{cha:datasets}

This chapter presents summary descriptions of the various data sets
that are relevant to this analysis and further discussion on how they
were manipulated in preparation for analysis.  Operations where
multiple data sets are used in conjunction are deferred to Chapter
\ref{cha:analysis}.

The general approach with the MLCT and NLCD data sets is to reclassify
their categories, calculate per-pixel, per-class areas at the native
resolutions, and aggregate the new classification to the 5$'$ grid.
The purpose of the reclassification is to reduce the number of classes
and have a uniform set of classes across data sets.  The challenge in
this is that classification defintions are sometimes subtly different
which makes direct comparison across data sets somewhat subjective, so
we describe the mapping between original and simplified
classifications.  We apply and aggregation operation that calculates
the relative proportion of each class in the new classification system
present in each 5$'$ grid cell according to the base data.  In this
process we convert classified maps whose pixels have discrete values
to a stack of maps, one map per class, whose pixels have real number
values on the interval $[0,1]$ representing fractional areas and are
constrained to sum to unity for each pixel through the stack.  In the
general case of the MLCT data product the process converts two
discrete, thematic variables and one continuous variable, those being
a primary covery type, a secondary cover type, and classification
confidence level respectively, into a set of continuous variables
representing fractional areas for the cover types in the siplified
classification system.  This general case is also compared to simpler
cases of the NLCD and considering only the primary classification of
MLCT.  In these cases the process is simplified by considering only a
primary thematic layer and performing the aggregation without a
secondary cover type or confidence level by which to relate them but
we are able to reuse the same functions for the raster calculations.

To illustrate the process of converting the these data sets from their
original representation we are including maps of an area of
southeastern Michigan to show greater detail through each step of the
process.  We chose this region for its diversity of land covers and
uses, its relative diveristy of agricultural commodities across its
significant cropland area, the significant presence of the mosaic
class to illustrate our method for its decomposition and its
familiarity to our principal author, being his brthplace.


\section{MODIS Land Cover Type (MLCT)}
\label{sec:mlct}

In preparation for this analysis we prepared the 2001 MLCT data by
patching together the tiles as delivered in the equal-area sinusoidal
projection, reprojecting that mosaic to geographic coordinates, and
extracting a subset for the conterminous United States (cUSA).  These
preparation steps were carried out in a \texttt{GRASS} database prior
to the adoption of the reproducible research framework for this paper,
so those steps are not demonstrated here.  The cUSA study area is
defined as the set of 5$'$ grid cells that intersect with the cUSA
polygon in version 1 of the Global Administrative Areas (GADM) vector
data set, which includes the water bodies on the American side of the
international border across the Great Lakes, but does not extend to
oceanic waters beyond the coastal grid cells that intersect with any
land mass.

In this section we will demonstrate the process of converting the MLCT
data from its native form, consisting of primary cover type,
classification confidence for the primary cover, and secondary
(alternate) cover type at 15$''$ resolution, to a stack of cover
fractions at 5$'$ resolution using the simplified cover/use
classification specified by the PEEL model.

\subsection{Reclassification}
\label{sec:mlct-reclass}

The following table shows the mapping of the IGBP classes used in the
original MLCT data to the simplified classification designed for the
PEEL model.

\missingfigure{MLCT reclassification table}





\begin{figure}[hpt] 
  \centering
  

\includegraphics{fig_thumb_pri_reclass}
%\end{center} 
\caption{MLCT primary cover reclassified detail} 
\label{fig:thumb_pri_reclass} 
\end{figure} 

\todo{Is this figure any better placed than others?}

\autoref{fig:thumb_pri_reclass} shows the result of reclassifying the MLCT data
for our detailed study area.  From this map we see that this area is
dominated by the crop class in the north and the mosaic class to the
south with scattered forests and pockets of development throughout.
The urban complex of Port Huron, Michicagn and Sarnia, Ontario is
visible in the southeast corner.  along with the confidence level given for
the primay classification.

\begin{figure}[hpt] 
\begin{center}
  

\includegraphics{fig_thumb_sec_reclass}
\end{center} 
\caption{MLCT secondary cover reclassified detail} 
\label{fig:thumb_sec_reclass} 
\end{figure} 


In \autoref{fig:thumb_sec_reclass} we notice that areas in the
northern and central sections of the map that were classified as crop
in the primary layer have null values in the secondary class.  It is
apparent that where a secondary class is given that the mosaic class
is often indicated where the primary class indicates cropland and vice
versa.  It is possible for primary and secondary classes to be
assigned to the same category because of the reclassification step.
When one of our pixels indiactes the forest class for both its primary
and secondary classifications it simply reflects a distinction between
sub-types of forest in the original data, for example evergreen and
deciduous.

\begin{figure}[hpt] 
\begin{center}
  

\includegraphics{fig_thumb_pct}
\end{center} 
\caption{MLCT primary cover classification confidence} 
\label{fig:thumb_pct} 
\end{figure} 

\autoref{fig:thumb_pct} shows the confidence level as a percentage.
We see that the areas where no secondary class is given are areas
where confidence is 100\% and the primary classification is cropland
and therefore would be accounted as 100\% cropland by area by any
method of adding up these areas.  In light of this observation it is
clear that MLCT will generally over-estimate cropland because it is
certain that these areas are not completely under cultivation but
rather are interspersed with homesteads, fence lines, small wood lots,
roads, and such cultural features.  In areas such as this that were
made available for settlement in the 19th century according to the
Public Land Survey System (PLSS) we expect to find roads delineating
every square mile in general.  

The relationships described among the three layers of the MLCT are
perhaps more easily appreciated visually by mapping the individual
classes separately.  \autoref{fig:thumb_pri_facet} does this for the
primary class in our example detail area and
\autoref{fig:thumb_sec_facet} for the secondary class.


\begin{figure}[hpt] 
\begin{center}
  

\includegraphics{fig_thumb_pri_facet}
\end{center} 
\caption{MLCT primary covers shown separately, detail} 
\label{fig:thumb_pri_facet} 
\end{figure} 


\begin{figure}[hpt] 
\begin{center}
  

\includegraphics{fig_thumb_sec_facet}
\end{center} 
\caption{MLCT secondary covers shown separately, detail} 
\label{fig:thumb_sec_facet} 
\end{figure} 

%\subsubsection{Analysis Area}
%\label{sec:reclass-analysis-area}

Conveniently we are able to reuse the same functions for
reclassification and mapping of the data that we have prepared for the
larger study area.  \autoref{fig:mlct_pri_reclass} shows the map of
the primary classification across the cUSA, and likewise
\autoref{fig:mlct_sec_reclass} for the secondary layer and
\autoref{fig:mlct_pct} for the confidence level.  Because the maps are
showing a greater extent in relatively the same amount of page space
it is even more useful to create the facet maps for the individual
classes as \autoref{fig:mlct_pri_facet} and
\autoref{fig:mlct_sec_facet} have done.  From these maps familiar
generalities of the cUSA's geography are more apparent, such as the
prevalence of forests in the east and northwest, cropland in the
midwest, shrub lands in the southwest and open lands across the west.
It is interesting to note that the mosaic class is primarily
concentrated in the eastern portion of the study area which we can
attribute to greater population density, topography, and historical
patterns of settlement resulting in characteristically smaller parcels
and a greater degree of mixing among agricultural uses and natural
covers.



\begin{figure}[hpt] 
\begin{center}


\includegraphics{fig_mlct_pri_reclass_trim}
\end{center} 
\caption{MLCT primary cover reclassified} 
\label{fig:mlct_pri_reclass} 
\end{figure} 


\begin{figure}[hpt] 
\begin{center}
  

\includegraphics{fig_mlct_sec_reclass_trim}
\end{center} 
\caption{MLCT secondary cover reclassified} 
\label{fig:mlct_sec_reclass} 
\end{figure} 


\begin{figure}[hpt] 
\begin{center}
  

\includegraphics{fig_mlct_pct_trim}
\end{center} 
\caption{MLCT primary cover classification confidence} 
\label{fig:mlct_pct} 
\end{figure} 


\begin{figure}[hpt] 
\begin{center}
  

\includegraphics{fig_mlct_pri_facet}
\end{center} 
\caption{MLCT primary covers shown separately} 
\label{fig:mlct_pri_facet} 
\end{figure} 


\begin{figure}[hpt] 
\begin{center}
  

\includegraphics{fig_mlct_sec_facet}
\end{center} 
\caption{MLCT secondary covers shown separately} 
\label{fig:mlct_sec_facet} 
\end{figure} 


\subsection{Aggregation}
\label{sec:mlct-aggr}

MLCT has a nominal resolution of 500m which roughly equates to 15$''$
at the eqautor and so is conveniently an even division of the 5$'$
grid to which we wish to aggregate it, the two related by a factor of
20.  Therefore each cell in the output of this aggregation will be a
function of the 400 orginal MLCT pixels within its footprint.  The
dataset consists of a primary classification, along with a measure of
confidence up to 100\%, and a secondary classification.  The secondary
cover type is given as the most likely alternative to the primary type
\citep{Friedl2010}, but for purposes of our analysis we are taking a
more probabilistic view and incorporating all available information
from the base data.  Because we are aggregating the data up to
5-arcmin resolution there is no expectation that the sub-pixel
fractions at full resolution are spaitally specific, but in the
aggregate our characterization of each grid cell's composition will be
nuanced by this additional information.  The primary class covers at
least roughly 50-60\% of a given pixel $x$, and this percent is almost
certainly a monotonically increasing function of the confidence
measure $c$.  \todo{cite email from Friedl}.  For the purposes of this
analysis we assume that this dependence is linear. Thus, for the
primary and secondary cover types in a pixel:

$$
A_p(x) = A_{min} + (1 - A_{min}) c(x)
$$
$$
A_s(x) = 1 - A_p(x)
$$

where $0.50 \le A_{min} \le 0.60$ is primarily chosen based on an
interpretation of $c$.  Given that there are only a handful of
examples of $c < 0.20$ \todo{consider including histograms showing
  confidence distribution}, setting $A_{min} = 0.50$ is appropriate.
Certainly for a classification to be considered the primary it must
represent a bare majority of the area covered by that pixel at
minimum, and the distributions of confidences indicate that the vast
majority of pixels contain greater than 60\% of their area in the
primary under the rubric described above.  The equations are simplified
as follows by assuming this value for $A_{min}$.

$$
A_p(x) = \dfrac{1 + c}{2}
$$
$$
A_s(x) = 1 - A_p(x) = \dfrac{1-c}{2}
$$

% will this help?



Applying these formulae results in a map for each cover type where the
pixel values are the sub-pixel areas on the interval $[0,1]$.  The map
of the fraction of the primary cover type is visually equivalent to
that of the classification confidence level because the former is
simply a linear scaling and offset of the latter.  \autoref{fig:thumb_fracs} shows the result of calculating $A_p + A_s$ for each individual class.  

%\subsubsection{Detail Area}
%\label{sec:agg-detail-area}


\begin{figure}[hpt] 
\begin{center}
  

\includegraphics{fig_thumb_fracs}
\end{center} 
\caption{Sub-pixel fractions at original resolution for $A_{min}=0.5$}
\label{fig:thumb_fracs}
\end{figure} 

By way of comparison we also consider the trivial case of settimg
$A_{min} = 1$ which indicates that the secondary cover is ignored
altogether and the primary cover is taken to represent 100\% of the
pixel area.  \autoref{fig:thumb1_fracs} shows these difference.  The
effect of adjusting $A_{min}$ is subtle; we will examine it more
closely after aggregating to the 5$'$ grid.

\begin{figure}[hpt] 
\begin{center}
  

\includegraphics{fig_thumb1_fracs}
\end{center} 
\caption{Sub-pixel fractions at original resolution for $A_{min}=1$}
\label{fig:thumb1_fracs}
\end{figure} 

Computationally the process of converting the reclassified maps to
sub-pixel fractions at the desired 5-arcmin resolution is a three-step
process.  First we calculate the fraction of the primary cover type as
a function of the classification confidence as described above.  Next,
a sub-pixel fraction for each cover type is calculated at full
resolution, recognizing that the primary and secondary classes may be
identical after the reclassification, such as cases where the original
data indicated two different type of forests.  Aggregating to a
coarser resolution is a simple matter of calculating the mean of these
values over the intersecting pixels at the original resoution.
Because the desired 5$'$ resolution is a multiple of the original
15$''$ resolution the pixels are perfectly nested, which is
convenient for properly computing this mean.


\begin{figure}[hpt] 
\begin{center}
  

\includegraphics{fig_thumb_agg}
\end{center} 
\caption{Aggregated sub-pixel fractions for $A_{min}=0.5$}
\label{fig:thumb_agg}
\end{figure} 

\begin{figure}[hpt] 
\begin{center}
  

\includegraphics{fig_thumb1_agg}
\end{center} 
\caption{Aggregated sub-pixel fractions for $A_{min}=1$}
\label{fig:thumb1_agg}
\end{figure} 


Before proceeding further it is interesting to inspect the differences
between the aggregated maps for the chosen values of $A_{min}$ as
shown in \autoref{fig:thumb_agg_diff}.  Positive values indicate that
$A_{min} = 0.5$ resulted in a greater fraction.  The main message from
this chart is that considering the secondary cover class results in
greater mixture between the crop and mosaic classes because cropland
is reduced in the north of the detail area where it was dominant in
the primary land cover type, and simlarly for mosaic in the south.
The relative suitability of these choices for $A_{min}$ is discussed
in \autoref{cha:analysis}.



\autoref{fig:thumb_agg_diff} emphasizes the difference between the
choice of $A_{min}=0.5$ and $A_{min}=1.0$ for the calculation of the
sub-pixel fractions and their aggregation to 5$'$ with a difference
map.  Positive values in the map indicate areas where $A_{min}=0.5$
produced a greater value.  We see more clearly from this set of maps
that the effect of considering the secondary class results in a shift
of up to 10\% of total cell area from crop to mosaic in the north of
our detail area and vice versa for the southern portion.  This
decrease in the relative dominance of the primary class is expected as
we saw from the earlier maps (\autoref{fig:thumb_pri_facet} and
\autoref{fig:thumb_sec_reclass}) of the MLCT data which classes were
indicated by the secondary classes in those areas.

\begin{figure}[hpt]
\begin{center}
  
%def

\includegraphics{fig_thumb_agg_diff}
\end{center} 
\caption{ Difference of aggregated sub-pixel fractions}
\label{fig:thumb_agg_diff}
\end{figure} 

%\subsubsection{Analysis Area}
%\label{sec:agg-analysis-area}


We apply the same functions for calculating the 15$''$-resolution map
of the primary cover class as a function of the confidence level $c$
for the entire cUSA study area, converting those to per-class
fractions at the same extent and scale, and aggregating those values
to the 5$'$ grid.  The corresponding figures are not shown because the
decrease in relative resolution makes interpretation difficult.  Based
on the behavior that these functions exhibited over the detail area we
can be confident that they will perform correctly over the greater
extent.



\subsection{Mosaic decomposition}
\label{sec:decomposition}

The MLCT classification includes a type that is problematic for the
economic models for which this data set is intended, the ``cropland /
natural vegetation mosaic'' class.  This class is defined as a hybrid
of cropland and some mixture of natural covers (forest, shrub, or
open) with no single component exceeding 60\% \citep{Friedl2002} and
croplands generally comprising 40--60\% of pixel area \todo{cite Friedl
  email}. Being a hybrid of developed land use and natural land cover
we wish to differentiate the cropland from the natural vegetation in
order to calculate a more meaningful total for cropland area and
thereby eliminate the mosaic class from the final tabulation.  In the
present implementation of the reclassification and aggregation process
we are making three very simple assumptions about the composition of
area delineated as mosaic lands:

\begin{enumerate}
\item Mosaic land is 50\% cropland.
\item The other 50\% is a blend of forest, open, and shrub in
  proportion to the expression of those classes in the same 5-minute
  cell.
\item In the absence of such information we simply assume that the
  natural component of the mosaic is an equal blend of all three.
\end{enumerate}

The intention here is to make simplifying assumptions that will allow
us to proceed with the evaluation of this analytical framework.
Although it may be interesting to vary the proportion used to
calculate the proportion of mosaic land to be allocated to crop land
we have no principled basis for this as of yet, considering that the
defintion implies that this proportion is variable across the MLCT
rather than being some unknown single-valued quantity.  The choice of
the 50\% level reflects the assertion that the mosaic is a cultural
class grouped with cropland and urban in the IGBP classification
scheme without overstating the degree of development.  MLCT provides
adequate variability in this dimension by commonly pairing cropland
and mosaic in the primary/secondary class data.  The second assumption
imposes that 15$''$ mosaic cells' non-crop portion will have the same
relative composition of forest, open, and shrub as the as the
non-mosaic portion of the 5$'$ grid cell in which it falls. Therefore
mosaic pixels in a 5$'$ cell where only forest is found of the three
non-crop mosaic components will be allocated 50\% crop and 50\%
forest.  \autoref{fig:thumb_nomos} and \autoref{fig:thumb1_nomos} show
the effect of decomposing the mosaic class in this fashion for
$A_{min}$ values of 0.5 and 1.0 respectively.



\begin{figure}[hpt]
\begin{center}
  


\includegraphics{fig_thumb_nomos}
\end{center} 
\caption{Aggregated cover fractions after mosaic decomposition, $A_{min}=0.5$}
\label{fig:thumb_nomos}
\end{figure} 

\begin{figure}[hpt]
\begin{center}
  


\includegraphics{fig_thumb1_nomos}
\end{center} 
\caption{Aggregated cover fractions after mosaic decomposition, $A_{min}=1.0$}
\label{fig:thumb1_nomos}
\end{figure} 



\begin{figure}[hpt]
\begin{center}
  

\includegraphics{fig_thumb_nomos_diff}
\end{center} 
\caption{Differences of sub-pixel fractions after mosaic
  decomposition, positive when $f(A_{min} = 0.5)$ is greater}
\label{fig:thumb_nomos_diff}
\end{figure} 





Our hypothesis from the outset is that there is information worth
capturing in the secondary class and classification confidence level
provided by MLCT.  We will test this hypothesis in
\autoref{cha:analysis} but in order to do so we need an ``observed
truth'' to provide an independent standard by which to make a
comparison on the basis of overall reduction in error at the 5$'$ grid
cell level.  The following section describes such a data set which
will be held up against these MLCT-derived data sets in the next
chapter.

% more appropriately addressed in following chapter

% Both 5-arcmin data sets derived from the MLCT in this fashion
% overestimate cropland area relative to that indicated by Agland2000,
% but the $A_{min} = 0.5$ variant better portrays the spatial variation
% judging from a simple root-mean-squared-error (RMSE)
% test. \todo{illustrate/demonstrate the RMSE test on the 5-arcmin MLCT
%   data sets}

\section{Agricultural Lands in the Year 2000 (Agland2000)}
\label{sec:agland2000}


The data set described by \citet{Ramankutty2008}, referred to in this
paper as ``Agland2000'', is the product of an effort to merge
satellite-derived LULC classifications with census data of
agricultural actviity compiled at national or sub-national levels
according to availability on around the turn of the last century.  It
uses both an older version of the MLCT (known as BU-MODIS) and the
GLC2000 data set mentioned in \autoref{sec:background} and a mask
based on climatic criteria and delineations of protected areas to
allocate the census data to the 5$'$ grid for both cropland and
pasture.  The ``open'' class in this data set has been renamed from
``pasture'' in its creator's nomenclature, but it is clear from its
ditribution shown in \autoref{fig:agland} that it represents a
phenomena that is not apparent in the MLCT data, so we do not attempt
to use it or reconcile it here, rather only carry it along to a small
degree for sake of comparison.  We attribute this discrepancy to
commingling of managed pasture lands and natural open land in the MLCT
classification.  It is important to note that Agland2000 is used as an
input into the classification algorithm of the version of MLCT that we
are using here and acknowledge the possibility of circularity when
comparing the two, but because of its basis in census data we will use
the cropland component of Agland2000 as an ``observed truth'' for the
purposes of evaluating our incremental adjustments to the maps we
derive from MLCT in \autoref{cha:analysis}.
  
%def 

\begin{figure}[hpt]
\begin{center}
  

\includegraphics{fig_thumb_agland}
\end{center} 
\caption{Agland2000 distribution in detail area}
\label{fig:thumb_agland} 
\end{figure} 

\begin{figure}[hpt]
\begin{center}
  

\includegraphics{fig_agland_trim}
\end{center} 
\caption{Agland2000 distribution in cUSA study area}
\label{fig:agland} 
\end{figure} 


\begin{comment}
\section{Major Land Uses (MLU)}
\label{sec:mlu}

This is a tabular data set published by the Economic Research Service
(ERS) at the USDA of land acreages by various uses and covers at a
state level.  We hope to compare our results to this data on a
state-by-state basis in order as a check and possibly incorporate some
of its information as a refinement.

\end{comment}

\section{National Land-cover Database 2001 (NLCD)}
\label{sec:nlcd}

\citet{Homer2004}


The NLCD gives a higer-resolution (30m) snapshot of LULC circa 2001.
\todo{check whether/how urban, water, wetland are informed with priors
  in NLCD} Reclassifying and aggregating this data to 5-arcmin
resolution in a fashion similar to that used for the MLCT is expected
to give better estimations of aggregate area for detailed features
like rural transportation networks and small stream and wetland
features.  This will compensate for MLCT's bias against these finely
detailed structures due to it's resolution.  It is the availability of
this information that makes it difficult to apply this analysis beyond
the United States without access to a comparable data set with global
extents.  The analysis is restricted to the conterminous US because of
the relative paucity of agricultural activity in Hawaii and Alaska.

As with the MLCT the process of reclassification and aggregation is
performed for both the detail region and the complete region.

One limitation of the \texttt{raster} library for \texttt{R} that we
are using is that the aggregation function requires that the output
resolution be a multiple of the output resolution.  The 30m resolution
of the NLCD equates to 1.25361$''$ and so does not satisfy this
requirement.  This deficiency was addressed by resampling the input to
1.25$''$ resolution prior to export from \texttt{GRASS} for this
analysis using a nearest-neighbor sampling algorithm, which gives an
even factor of 240 between the two resolutions.

\todo{Incorporate Joshua's suggestion to show further NLCD detail to better illustrate the discrepancy in developed areas}


\subsection{Reclassification}
\label{sec:nlcd-reclass}

\missingfigure{NLCD reclassification table}


\begin{figure}[hpt] 
\begin{center}


\includegraphics{fig_thumb_nlcd_reclass}
\end{center} 
\caption{NLCD reclassified} 
\label{fig:thumb_nlcd_reclass} 
\end{figure} 

\begin{figure}[hpt] 
\begin{center}
  

\includegraphics{fig_thumb_nlcd_facet}
\end{center} 
\caption{NLCD covers shown separately, detail} 
\label{fig:thumb_nlcd_facet} 
\end{figure} 

\subsection{Aggregation}
\label{sec:nlcd-aggr}

The same code used for refactoring the MLCT when considering only the
primary cover type can be applied here.

Repeating this process for the entire study area is computationally
expensive due to the NLCD's high resolution.


 

\begin{figure}[hpt] 
\begin{center}
  


\includegraphics{fig_thumb_nlcd_agg}
\end{center} 
\caption{NLCD aggregated cover fractions, detail area}
\label{fig:thumb_nlcd_agg}
\end{figure} 




\begin{figure}[hpt] 
\begin{center}
  


\includegraphics{fig_nlcd}
\end{center} 
\caption{NLCD aggregated cover fractions}
\label{fig:nlcd}
\end{figure} 

\begin{comment}
\section{Cropland Data Layer (CDL)}
\label{sec:cdl}

\missingfigure{Table or chart showing CDL covereage for various years}

The CDL is only available for a small number of states in 2001.  If
time allows it might be good to compare what is available with our
results as another independent evaluation against a higher-resolution
data set.

\subsection{Reclassification}
\label{sec:cdl-reclass}



%  gdalbuildvrt -tr 0.0002777778 0.0002777778 -te -124.8333 24.5 -66.91667 49.33333 cdl_2001.vrt $(find . -name "*2001*")


\todo{Calculate CDL mask for 5-arcmin cells completely filled}
\todo(Calculate CDL aggregated in GRASS}




\missingfigure{CDL reclassification table}

\subsection{Aggregation}
\label{sec:cdl-aggr}
\end{comment}

\section{Harvested Area and Yields of 175 Crops (175crops2000)}
\label{sec:175crops2000}

\citet{Monfreda2008}

\missingfigure{Table of crops and types reproduced from \citep{Monfreda2008}}

\missingfigure{Summary table of crop aggregations for our model}

\todo{Address issue of smaller land mask for 175crops2000 and Agland2000}

This data set will provide the information needed to disaggregate the
cropland area taken from Agland2000.  It is not possible to use this
data directly because it reflects only harvested area and so ignores
various types of ancillary agricultural land, rather it will provide
proportions for the disaggregation at the grid cell level.  Rather
than considering the full array of 175 crops we will consider only
corn, soy, wheat, rice, and sugarcane individually, combine other
cereals into their own class, and combine all remaining crops as a
catch-all ``other'' category.  Field crops will be distinguished from
orchard / plantation crops that would likely fall under areas
classified by MLCT as forest or shrub in this step.


\begin{figure}[hpt] 
\begin{center} 


\includegraphics{fig_crops}
\end{center} 
\caption{175Crops2000 category maps} 
\label{fig:crops} 
\end{figure} 


%%% Local Variables: 
%%% mode: latex
%%% TeX-master: "thesis"
%%% End: 

% -*- mode: noweb; noweb-default-code-mode: R-mode; -*-


%\SweaveOpts{ eps=FALSE, pdf=FALSE, png=TRUE }

\graphicspath{ {analysis/} }

\chapter{Analysis}
\label{cha:analysis}


% latex table generated in R 2.11.1 by xtable 1.5-6 package
% Mon Dec  6 14:40:52 2010
\begin{table}[ht]
\begin{center}
\begin{tabular}{rrrrrrrrrr}
  \hline
 & crop & open & barren & forest & shrub & urban & water & wetland & mosaic \\ 
  \hline
As00 & 369.6 & 545.8 & 28.9 & 353.6 & 341.8 & 29.8 & 75.0 & 11.0 & 237.0 \\ 
  As05 & 379.2 & 516.3 & 32.9 & 344.5 & 359.1 & 27.3 & 73.4 & 26.1 & 232.8 \\ 
  ag & 446.5 & 557.1 &  &  &  &  &  &  &  \\ 
  agc & 450.1 & 558.3 & 25.3 & 450.3 & 368.0 & 35.1 & 74.5 & 30.9 &  \\ 
  nlcd & 310.8 & 429.6 & 24.5 & 513.2 & 420.1 & 102.8 & 96.5 & 95.0 &  \\ 
   \hline
\end{tabular}
\caption{Total Acreages by Map and Cover}
\label{tab:total}
\end{center}
\end{table}
The MLCT indicates 379.2Ma (153.4Mha) of
cropland in the cUSA in 2001. Assuming that 50\% of the
cropland/natural vegetation mosaic is additional cropland area gives
an additional 116.4Ma (47.1Mha)
of agricultural land. This gives a total area of 
495.6Ma (200.6Mha)
of total area directly associated with
agricultural activity according to the IGBP classification used in the
MLCT.

Aglands2000 indicates roughly 446.5Ma (180.7Mha) 
of cropland.

Pasture indicated by Aglands2000 appears to be a broader
classification than that of the NLCD's pasture class because much of
the grazing land east of the Mississippi river counted in the
Aglands2000 pasture map is absent in the NLCD pasture class.

Due to its greater resolution (30m) the NLCD is better suited at
discerning developed areas in rural landscapes ranging from rural
roads to farmsteads to small communities that do not show up in the
MLCT data. There is a total area of roughly 74 Ma (30 Mha) of
development remaining after subtracting the MLCT urban class from all
developed classes in the NLCD where the NLCD shows greater development
after they have both been aggregated to the 5-arcmin grid. Applying
this area as an offset to the cropland area in Aglands2000 brings us
closer to the expected acreage under cultivation in 2001, although
this assumes that all of that development intersects with MLCT
cropland area.


% latex table generated in R 2.11.1 by xtable 1.5-6 package
% Mon Dec  6 14:44:54 2010
\begin{table}[ht]
\begin{center}
\begin{tabular}{rrrrr}
  \hline
 & rmse\_frac & bias\_frac & rmse\_acres & bias\_acres \\ 
  \hline
barren & 0.05 & 2.68E-04 & 902 & 7 \\ 
  crop & 0.15 & 6.94E-02 & 2515 & 1142 \\ 
  forest & 0.20 & -2.82E-02 & 3249 & -516 \\ 
  open & 0.32 & 6.13E-02 & 5418 & 1056 \\ 
  shrub & 0.33 & -2.65E-02 & 5572 & -427 \\ 
  urban & 0.06 & -3.34E-02 & 973 & -555 \\ 
  water & 0.04 & -1.11E-02 & 580 & -181 \\ 
  wetland & 0.10 & -3.16E-02 & 1608 & -526 \\ 
   \hline
\end{tabular}
\caption{Errors and Biases of Aglands Complete relative to NLCD}
\label{tab:ebagc}
\end{center}
\end{table}% latex table generated in R 2.11.1 by xtable 1.5-6 package
% Mon Dec  6 14:44:54 2010
\begin{table}[ht]
\begin{center}
\begin{tabular}{rrrrr}
  \hline
 & rmse\_frac & bias\_frac & rmse\_acres & bias\_acres \\ 
  \hline
barren & 0.07 & 1.95E-03 & 1140 & 36 \\ 
  crop & 0.19 & 3.18E-02 & 2945 & 482 \\ 
  forest & 0.20 & -7.77E-02 & 3380 & -1309 \\ 
  open & 0.33 & 5.84E-02 & 5428 & 953 \\ 
  shrub & 0.29 & -4.34E-02 & 4783 & -642 \\ 
  urban & 0.06 & -3.62E-02 & 988 & -598 \\ 
  water & 0.03 & -1.09E-02 & 504 & -177 \\ 
  wetland & 0.10 & -4.15E-02 & 1759 & -689 \\ 
   \hline
\end{tabular}
\caption{Errors and Biases of MLCT, $A_s = 0.0$ relative to NLCD}
\label{tab:ebmlct00}
\end{center}
\end{table}% latex table generated in R 2.11.1 by xtable 1.5-6 package
% Mon Dec  6 14:44:54 2010
\begin{table}[ht]
\begin{center}
\begin{tabular}{rrrrr}
  \hline
 & rmse\_frac & bias\_frac & rmse\_acres & bias\_acres \\ 
  \hline
barren & 0.06 & 3.89E-03 & 1092 & 69 \\ 
  crop & 0.17 & 3.63E-02 & 2690 & 560 \\ 
  forest & 0.20 & -8.21E-02 & 3362 & -1383 \\ 
  open & 0.30 & 4.35E-02 & 4891 & 711 \\ 
  shrub & 0.27 & -3.43E-02 & 4410 & -500 \\ 
  urban & 0.06 & -3.74E-02 & 1027 & -619 \\ 
  water & 0.03 & -1.17E-02 & 529 & -190 \\ 
  wetland & 0.10 & -3.42E-02 & 1595 & -565 \\ 
   \hline
\end{tabular}
\caption{Errors and Biases of MLCT, $A_s = 0.5$ relative to NLCD}
\label{tab:ebmlct05}
\end{center}
\end{table}

\begin{figure} 
\begin{center} 
\includegraphics{fig_agc}
\end{center} 
\caption{Aglands Complete cover maps} 
\label{fig:agc} 
\end{figure} 

\begin{figure} 
\begin{center} 
\includegraphics{fig_nlcd}
\end{center} 
\caption{NLCD cover maps} 
\label{fig:nlcd} 
\end{figure} 

\begin{figure} 
\begin{center} 
\includegraphics{fig_diff}
\end{center} 
\caption{Difference maps, Aglands Complete minus NLCD} 
\label{fig:diff} 
\end{figure} 

\begin{figure} 
\begin{center} 
\includegraphics{fig_cordiff}
\end{center} 
\caption{Correlations across cover type in difference maps} 
\label{fig:cordiff} 
\end{figure} 

The elements of the matrix have been reordered according to the
clustering forumla given in \citet[sec. 6.2.3]{Sarkar2008} in order to
achieve a degree of visual clustering among the correlation vectors.

%%% Local Variables: 
%%% mode: latex
%%% TeX-master: "thesis"
%%% End: 


\chapter{Conclusions}
\label{cha:conclusions}

The goal of this study was to produce a LULC data set that was as
accurate as possible with respect to its representation of the
distribution of agricultural production and that also offers a
reasonable characterization of non-crop covers and land use beyond
those agricultural uses.  We have accomplished that.  In doing so we
have adopted a sub-pixel data structure for conveying land use and
land cover information, albeit at a low spatial resolution by today's
remote sensing and GIS standards.  However, we hope that our readers
will consider that this data structure has particular mathematical
properties that may make it useful for their applications.  The
ability to perform raster algebra on these stacks of maps made it
possible to apply scaling factors and offsets with concise syntax in
the \texttt{R} language that would not have been feasible in a
discrete, categorical framework.

We maintain that the reproducible research aspect of this study was
critical to its success.  By elaborating the analysis in an
interactive environment where every component of every data structure
is subject to inspection many missteps were discovered in the course
of our work.  In other GIS analysis environments it might have been
too difficult to perform basic sanity checks on intermediate outputs
before moving on and too easy to rely on a sense of ``everything looks
right'' in a GUI environment.  The ability to point to a body of
source code that expresses the steps of the analysis is a poignant
example of reproducibility, a pillar of the scientific method that has
fallen out of vogue because of the complexity of modern spatial
analysis until the recent emergence of applicable software tools
coupled with adequate computing resources.

This analysis would not have been possible without the \texttt{raster}
package for \texttt{R} developed by Hijmans, van~Etten and other
contributors.  We consider this interface to geospatial raster data
sets in the \texttt{R} statistical analysis environment an important
contribution to spatial analysis and a laudable accomplishment because
it unleashes the power of a sophisticated, popular, open-source, free
software programming language for statistical operations on large
geospatial raster data sets.  We expect that this demonstration will
foster additional interest in and use of this software, as well as
contribution to its continuing development.  Directions for possible
enhancement that would increase the utility of this package based on
our experience include streamlining a truly functional programming
interface by improving on the existing \texttt{overlay()} function,
harnessing available \texttt{R} extensions for parallelism and porting
core functionality to a \texttt{C} or \texttt{C++} library to improve
performance, and improving the visualization interface, perhaps
through application of the \texttt{ggplot2} package as we have done
here.

In the CIM-EARTH/PEEL research agenda the next logical extension of
the envisioned land use transition model would be to apply it to a
global study area.  Once a method for creating a global initialization
data set for the year 2001 is formulated we would like to apply that
method to the subsequent years of the MLCT time series.  In order to
do so we would need ancillary data that spatially extends the aspects of the
NLCD that we have employed and temporally extends the information
given in the Ramankutty \& Monfreda data sets.  At this time a method
for calculating the correction offsets for over-estimation of cropland
and under-estimation of inland water features, wetlands, and
development/transportation infrastructure over a wider area with
greater time depth has not been identified.  Only with this
information in hand were we able bring MLCT cropland and Agland2000
cropland into close enough agreement to minimize the mathematical
manipulation necessary to reasonably quantify the non-crop cover and
use classes with the desired fidelity to the best-available rasterized
agricultural census data.

In the absence of high-resolution data on rural development, being the
low-density portion of PEEL's ``urban'' class that falls below MLCT's
detection threshold we propose that it might be possible to model the
over-estimation factor of the MLCT cropland class.  This factor would
be defined as the ratio of total area encompassed by the MLCT cropland
classification to acreage actually under cultivation and could
potentially be modeled as a function of classification confidence and
secondary class using the data described and produced here as a
training set.  The null hypothesis in the formulation of such a model
is that there is enough diversity among agricultural landscapes in our
cUSA study area to adequately characterize agricultural landscapes
around the world in this regard.  Similarly it might be possible to
directly model the ``urban'' percentage below the MLCT detection
threshold as a function of population density and agricultural
productivity, identifying said threshold in the process.  There is a
clear dependency between these offsets in agriculturally productive
regions so modeling them in conjunction somehow may be
constructive.  We expect that global offsets for the water and
wetland classes will be harder to obtain without corresponding proxy
statistics with which to formulate a model but perhaps we can expect
greater availability of spatially explicit catalogs of ecological
services and sensitive/protected areas in the near future that would
close these gaps in the available information.

\todo{Should we elaborate on this validation exercise in the conclusion or just leave it out?}

Comparison of this data to other available LULC characterizations,
particularly the Major Land Use (MLU) data and the Cropland Data Layer
(CDL) from the USDA, would provide useful validation metrics.


\backmatter

\bibliography{thesis}

\appendix

\todo{Clean up source code by removing commented code that is no
  longer useful}
\todo{Fix problems with appendix headings}


\chapter{Source Code: Data Sets}

\section*{R helper functions}
\lstinputlisting{code/peel.R}

\section*{R code embedded in the chapter}
\lstinputlisting{datasets.R}

\chapter{Source Code: Analysis}

\section*{R helper functions}
\lstinputlisting{code/analysis.R}

\section*{R code embedded in the chapter}
\lstinputlisting{analysis.R}

\end{document}