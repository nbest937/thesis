\documentclass[draftthesis]{neiuthesis} 

\usepackage{amsmath}
\usepackage{natbib}
\usepackage[prependcaption]{todonotes} %[obeyDraft]
\usepackage{Sweave}
% chages to Sweave.sty defaults
% from http://faculty.agecon.vt.edu/moeltner/AAEC5126/Sweave/docs/Arnholt_tutorial.pdf
\DefineVerbatimEnvironment{Sinput}{Verbatim}{fontsize=\small,fontshape=n}
\DefineVerbatimEnvironment{Soutput}{Verbatim}{fontsize=\small,fontshape=n}
\DefineVerbatimEnvironment{Scode}{Verbatim}{fontsize=\small,fontshape=n}
\usepackage{graphicx}
\usepackage{listings} 
\usepackage{courier} 
\lstset{breaklines=true,basicstyle=\ttfamily,language=R}
\usepackage{pdfpages}
\usepackage{pbox}
\usepackage{verbatim}
\usepackage{rotating}
\usepackage[normalem]{ulem}
\usepackage{hyperref}
\hypersetup{
  pdftitle={NEIU G\&ES MA Thesis},%
  pdfauthor={Neil Best <nbest@alum.mit.edu>},%
  pdfsubject={land use / land cover},%
  pdfkeywords={agriculture}{United States},%
  pdffitwindow=true,%
  pdfstartview={FitH}
}
\usepackage[all]{hypcap}

\begin{document} 
\bibliographystyle{chicago}

\title{Synthesis of a complete land use\slash land cover data set for the
  conterminous United States emphasizing accuracy in area and
  distribution of agricultural activity}
 

\othermasters{Master of Arts}{M.A.}  
\department{Geography \& Environmental Studies}

\author{Neil A. Best}
\degreeyear{August~2011}



\maketitle

%\includepdf[pages={2,1}]{neiuForms.pdf}
%!pdfTeX error: pdflatex (file ./neiuForms.pdf): PDF inclusion: invalid font in reference type <dictionary>

\frontmatter

\begin{abstract} This paper presents an effort to produce a new land
  cover data set for the conterminous United States of America (cUSA)
  that augments available agricultural land use data with other uses
  and natual covers to create a complete landscape characterization.
  Using the Agland2000 data set as a benchmark we formulate a
  hybridization of the MODIS Land Cover Type (MLCT) for 2001 and the
  2001 National Land Cover Database (NLCD) that is particularly
  tailored to serve as an initialization data set for long-term
  economic land use change models.  In order strike a balance between
  spatial precision and local diversity of use and cover the new data
  set has lower resolution than the MLCT (5$'$ vs. 500m) but
  represents consituent cover classes as sub-pixel fractions rather
  than discrete categories.  After aggregating to the 5$'$ grid we
  present a method for decomposing the natural vegetaion/cropland
  mosaic class found in MLCT into constituent classes as a function of
  the local landscape and quantify its contribution to aggregate
  acreages by class, particularly cropland.  We compensate for the
  absence of certain fine-grained details from MLCT, such as rural
  transportation networks, small settlements, linear water features,
  and wetlands, mainly due to sensor resolution, by incorporating
  corresponding components of the NLCD, after similar reclassification
  and aggregation, as a set of offsets to the MLCT-derived fractions.
  The 175Crops2000 data set, valuable for its basis in per-crop
  agricultural production statistics, is used as a guide to further
  decompose the cropland areas into a set of crop-specific
  sub-categories designed to facilitate the economic modeling goals of
  the simulations that will be initialized by this data product.  The
  final classification scheme, now conceptually equivalent to a stack
  of spectral bands with the additional quality that the components of
  each pixel sum to unity, is a mixture of a simplified version of the
  IGBP schema used in MLCT and a disaggregation of the monolithic
  cropland class that differentiates among the world's major commodity
  crops. \todo{It remains to be seen how well the disaggregation of
    commodities will work. We may have to leave it out.}  At each step
  of refinement we show that overall spatial distribution of cropland
  across the study area improves relative to the Aglands2000 data set.
  We close with a discussion of how this method might be applied
  globally and to successive years in the MLCT time series.
\end{abstract}


\chapter*{Acknowledgements}

\noindent This thesis is dedicated to my son, Leo.  Son, I began
working on this degree before you were born and my commitment to
completing it was sustained by my desire to demonstrate to you that in
life we finish what we have started.

\vspace{12pt}
\noindent I could not have completed this paper over the past year
and, by extension, my degree over more years than I care to mention
without the support of my loving wife, Laura.

\vspace{12pt}
\noindent I want to thank Dr. Nicholas Kouchoukos of Lanworth,
Inc. for throwing me in the deep end of applying the open-source
geospatial software tool chain to spatial analysis of agriculture.

\vspace{12pt}
\noindent This work was made possible through the support of my
employer, the Computation Institute at the University of Chicago, and
its director, Dr. Ian Foster under the Community Integrated Model of
Economic and Resource Trajectories for Humankind project (CIM-EARTH,
\url{http://www.cim-earth.org/}).

\vspace{12pt}
\noindent My thesis committee was comprised of Dr. Monika Mihir
(chair), Dr. Erick Howenstine (department head), both of the
Department of Geography \& Environmental Studies at Northeastern
Illinois University, and Dr. Joshua Elliott from the Computation
Institute.  I deeply appreciate their guidance and support through all
stages of this project.


\tableofcontents 
\listoftables
\listoffigures

%% Create a List of Abbreviations. The left column %% is 1 inch wide and left-justified
\chapter{List of Abbreviations}

\begin{symbollist*}
\item[175Crops2000] Harvested Area and Yields of 175 crops (M3-Crops
  Data) \citep{Monfreda2008}
\item[Agland2000] Agricultural Lands in the Year 2000 (M3-Cropland and
  M3-Pasture Data) \citep{Ramankutty2008}
\item[arcmin] minute of arc, 1/60th of a degree
\item[arcsec] second of arc, 1/60th of a minute, 1/3600th of a degree
\item[AVHRR] Advanced Very High Resolution Radiometer
\item[cUSA] conterminous (contiguous) Unites States of America, the ``lower 48''
\item[GLC2000] Global Land Cover 2000 \citep{EC2003}
\item[GRASS] Geographic Resources Analysis Support System, \url{http://grass.osgeo.org}
\item[IGBP] International Geosphere-Biosphere Programme
\item[LULC] land use / land cover
\item[MODIS] Moderate Resolution Imaging Spectroradiometer
\item[MLCT] MODIS Land Cover Type \citep{MLCT}
\item[NLCD] National Land-Cover Database, 2001 \citep{Homer2004}
\item[RMSE] root of the mean squared error
\item[SPAM] Spatial Production Allocation Model
\item[SPOT] Syst\`eme pour l'Observation de la Terre

\end{symbollist*}

%% Create a List of Symbols. The left column 
%% is 0.7 inch wide and centered
\chapter{List of Symbols}

\begin{symbollist}[0.7in]
\item[$A_{min}$] Minimum sub-pixel fraction possible for primary cover
  given in MLCT base data
\item[$A_s$] Sub-pixel fraction of secondary cover type, function of
  classification confidence level and $A_{min}$
\item[$A_p$] Sub-pixel fraction of primary cover type, function of
  classification confidence level and $A_{min}$
\item[$\hat\theta$] Predicted sub-pixel fraction
\item[$\theta$] Observed sub-pixel fraction
\item[$'$] minute of arc, 1/60th of a degree
\item[$''$] second of arc, 1/60th of a minute, 1/3600th of a degree
\end{symbollist}

\mainmatter

\todototoc
\listoftodos


\chapter{Introduction}
\label{cha:introduction}

\section{Objective}
\label{sec:objective}

Recent years have seen a significant increase in the availability of
global land cover data sets inclding the UMD Global Land Cover
Classification product of 1998 \citep{Hansen2000}, Global Land Cover
2000 (GLC2000) \todo{GLC2000 citation}, 
MODIS Land Cover Type (MLCT) \todo{MLCT citation}.  MLCT stands out
among these due to its spatial resolution, nominally 500m, and its
distinction as a time series rather than a snapshot.  Economic models
of land use and land conversion require information that describes a
complete, albeit simplified, description of land cover and
land-intensive econommic activity in order to produce meaningful
statements and predictions about the evolution of land use patterns.
``Complete'' in this context means that all cover types or uses for a
given portion of land area are assigned a category in the model.
However while MLCT does satisfy this condition of completeness it
presents two new complications that we must first address.

The first is that MLCT presents an embarassment of riches in terms of
detail.  Regardless of its classification accuracy, which is discussed
below, the 15-arcsecond resolution, nominally 500m, is simply too much
information to be able to run the economic models in a reasonable
amount of time even on world-class high-performance computing
platforms.  A current standard resolution for global models of many
types and global data sets is 5-arcminutes, which is equivalent to a
400:1 pixel count reduction.  Other data sets featured in this
analysis use this resolution which is convenient for formulation.

The second requirement for the new complete land cover data set that
we wish to produce is that it provide greater information regarding
agricultural activity.  MLCT presents a single class for cropland but
we wish to further disaggregate the areas of agricultural production
according to a few major commodities in order to incorporate greater
detail of agronomic and commercial factors into the models.  As we
will see, \citet{Monfreda2008} provides a wealth of data in this
regard by harvested area and yield for 175 crops globally, but does
not provide a complete land cover description.  

\section{Tools}
\label{sec:tools}

A secondary objective of this paper is to demonstrate the capabilities
of a set of open-source geospatial, analytical, and publishing
software that includes \href{http://www.gdal.org/}{GDAL},
\href{http://grass.osgeo.org/}{GRASS} \citep{GRASS},
\href{http://www.r-project.org/}{R} \citep{R} , and
\href{http://www.latex-project.org/}{\LaTeX} \citep{Lamport1994} .
The last two members of this list are bridged by
\href{http://www.stat.uni-muenchen.de/~leisch/Sweave/}{Sweave}
\citep{Leisch2002} which allows embedding of analytical code written
in the R language within a \LaTeX document so that one step towards
producing a publication-quality PDF is running the analysis and
injecting its results directly into the content of the paper,
including tables, charts, and maps.  The underlying analysis code will
appear as an appendix.  This is a demonstration of reproducible
research as described in \citet{Gentleman2007}.

s% -*- mode: noweb; noweb-default-code-mode: R-mode; -*-








\graphicspath{ {datasets/} }


\chapter{Data Sets}
\label{cha:datasets}

This chapter presents summary descriptions of the various data sets
that are relevant to this analysis and further discussion on how they
were manipulated in preparation for analysis.  Operations where
multiple data sets are used in conjunction are deferred to Chapter
\ref{cha:analysis}.

The general approach with the MLCT and NLCD data sets is to reclassify
their categories, calculate per-pixel, per-class areas at the native
resolutions, and aggregate the new classification to the 5$'$ grid.
The purpose of the reclassification is to reduce the number of classes
and have a uniform set of classes across data sets.  The challenge in
this is that classification defintions are sometimes subtly different
which makes direct comparison across data sets somewhat subjective, so
we describe the mapping between original and simplified
classifications.  We apply and aggregation operation that calculates
the relative proportion of each class in the new classification system
present in each 5$'$ grid cell according to the base data.  In this
process we convert classified maps whose pixels have discrete values
to a stack of maps, one map per class, whose pixels have real number
values on the interval $[0,1]$ representing fractional areas and are
constrained to sum to unity for each pixel through the stack.  In the
general case of the MLCT data product the process converts two
discrete, thematic variables and one continuous variable, those being
a primary covery type, a secondary cover type, and classification
confidence level respectively, into a set of continuous variables
representing fractional areas for the cover types in the siplified
classification system.  This general case is also compared to simpler
cases of the NLCD and considering only the primary classification of
MLCT.  In these cases the process is simplified by considering only a
primary thematic layer and performing the aggregation without a
secondary cover type or confidence level by which to relate them but
we are able to reuse the same functions for the raster calculations.

To illustrate the process of converting the these data sets from their
original representation we are including maps of an area of
southeastern Michigan to show greater detail through each step of the
process.  We chose this region for its diversity of land covers and
uses, its relative diveristy of agricultural commodities across its
significant cropland area, the significant presence of the mosaic
class to illustrate our method for its decomposition and its
familiarity to our principal author, being his brthplace.


\section{MODIS Land Cover Type (MLCT)}
\label{sec:mlct}

In preparation for this analysis we prepared the 2001 MLCT data by
patching together the tiles as delivered in the equal-area sinusoidal
projection, reprojecting that mosaic to geographic coordinates, and
extracting a subset for the conterminous United States (cUSA).  These
preparation steps were carried out in a \texttt{GRASS} database prior
to the adoption of the reproducible research framework for this paper,
so those steps are not demonstrated here.  The cUSA study area is
defined as the set of 5$'$ grid cells that intersect with the cUSA
polygon in version 1 of the Global Administrative Areas (GADM) vector
data set, which includes the water bodies on the American side of the
international border across the Great Lakes, but does not extend to
oceanic waters beyond the coastal grid cells that intersect with any
land mass.

In this section we will demonstrate the process of converting the MLCT
data from its native form, consisting of primary cover type,
classification confidence for the primary cover, and secondary
(alternate) cover type at 15$''$ resolution, to a stack of cover
fractions at 5$'$ resolution using the simplified cover/use
classification specified by the PEEL model.

\subsection{Reclassification}
\label{sec:mlct-reclass}

The following table shows the mapping of the IGBP classes used in the
original MLCT data to the simplified classification designed for the
PEEL model.

\missingfigure{MLCT reclassification table}





\begin{figure}[hpt] 
  \centering
  

\includegraphics{fig_thumb_pri_reclass}
%\end{center} 
\caption{MLCT primary cover reclassified detail} 
\label{fig:thumb_pri_reclass} 
\end{figure} 

\todo{Is this figure any better placed than others?}

\autoref{fig:thumb_pri_reclass} shows the result of reclassifying the MLCT data
for our detailed study area.  From this map we see that this area is
dominated by the crop class in the north and the mosaic class to the
south with scattered forests and pockets of development throughout.
The urban complex of Port Huron, Michicagn and Sarnia, Ontario is
visible in the southeast corner.  along with the confidence level given for
the primay classification.

\begin{figure}[hpt] 
\begin{center}
  

\includegraphics{fig_thumb_sec_reclass}
\end{center} 
\caption{MLCT secondary cover reclassified detail} 
\label{fig:thumb_sec_reclass} 
\end{figure} 


In \autoref{fig:thumb_sec_reclass} we notice that areas in the
northern and central sections of the map that were classified as crop
in the primary layer have null values in the secondary class.  It is
apparent that where a secondary class is given that the mosaic class
is often indicated where the primary class indicates cropland and vice
versa.  It is possible for primary and secondary classes to be
assigned to the same category because of the reclassification step.
When one of our pixels indiactes the forest class for both its primary
and secondary classifications it simply reflects a distinction between
sub-types of forest in the original data, for example evergreen and
deciduous.

\begin{figure}[hpt] 
\begin{center}
  

\includegraphics{fig_thumb_pct}
\end{center} 
\caption{MLCT primary cover classification confidence} 
\label{fig:thumb_pct} 
\end{figure} 

\autoref{fig:thumb_pct} shows the confidence level as a percentage.
We see that the areas where no secondary class is given are areas
where confidence is 100\% and the primary classification is cropland
and therefore would be accounted as 100\% cropland by area by any
method of adding up these areas.  In light of this observation it is
clear that MLCT will generally over-estimate cropland because it is
certain that these areas are not completely under cultivation but
rather are interspersed with homesteads, fence lines, small wood lots,
roads, and such cultural features.  In areas such as this that were
made available for settlement in the 19th century according to the
Public Land Survey System (PLSS) we expect to find roads delineating
every square mile in general.  

The relationships described among the three layers of the MLCT are
perhaps more easily appreciated visually by mapping the individual
classes separately.  \autoref{fig:thumb_pri_facet} does this for the
primary class in our example detail area and
\autoref{fig:thumb_sec_facet} for the secondary class.


\begin{figure}[hpt] 
\begin{center}
  

\includegraphics{fig_thumb_pri_facet}
\end{center} 
\caption{MLCT primary covers shown separately, detail} 
\label{fig:thumb_pri_facet} 
\end{figure} 


\begin{figure}[hpt] 
\begin{center}
  

\includegraphics{fig_thumb_sec_facet}
\end{center} 
\caption{MLCT secondary covers shown separately, detail} 
\label{fig:thumb_sec_facet} 
\end{figure} 

%\subsubsection{Analysis Area}
%\label{sec:reclass-analysis-area}

Conveniently we are able to reuse the same functions for
reclassification and mapping of the data that we have prepared for the
larger study area.  \autoref{fig:mlct_pri_reclass} shows the map of
the primary classification across the cUSA, and likewise
\autoref{fig:mlct_sec_reclass} for the secondary layer and
\autoref{fig:mlct_pct} for the confidence level.  Because the maps are
showing a greater extent in relatively the same amount of page space
it is even more useful to create the facet maps for the individual
classes as \autoref{fig:mlct_pri_facet} and
\autoref{fig:mlct_sec_facet} have done.  From these maps familiar
generalities of the cUSA's geography are more apparent, such as the
prevalence of forests in the east and northwest, cropland in the
midwest, shrub lands in the southwest and open lands across the west.
It is interesting to note that the mosaic class is primarily
concentrated in the eastern portion of the study area which we can
attribute to greater population density, topography, and historical
patterns of settlement resulting in characteristically smaller parcels
and a greater degree of mixing among agricultural uses and natural
covers.



\begin{figure}[hpt] 
\begin{center}


\includegraphics{fig_mlct_pri_reclass_trim}
\end{center} 
\caption{MLCT primary cover reclassified} 
\label{fig:mlct_pri_reclass} 
\end{figure} 


\begin{figure}[hpt] 
\begin{center}
  

\includegraphics{fig_mlct_sec_reclass_trim}
\end{center} 
\caption{MLCT secondary cover reclassified} 
\label{fig:mlct_sec_reclass} 
\end{figure} 


\begin{figure}[hpt] 
\begin{center}
  

\includegraphics{fig_mlct_pct_trim}
\end{center} 
\caption{MLCT primary cover classification confidence} 
\label{fig:mlct_pct} 
\end{figure} 


\begin{figure}[hpt] 
\begin{center}
  

\includegraphics{fig_mlct_pri_facet}
\end{center} 
\caption{MLCT primary covers shown separately} 
\label{fig:mlct_pri_facet} 
\end{figure} 


\begin{figure}[hpt] 
\begin{center}
  

\includegraphics{fig_mlct_sec_facet}
\end{center} 
\caption{MLCT secondary covers shown separately} 
\label{fig:mlct_sec_facet} 
\end{figure} 


\subsection{Aggregation}
\label{sec:mlct-aggr}

MLCT has a nominal resolution of 500m which roughly equates to 15$''$
at the eqautor and so is conveniently an even division of the 5$'$
grid to which we wish to aggregate it, the two related by a factor of
20.  Therefore each cell in the output of this aggregation will be a
function of the 400 orginal MLCT pixels within its footprint.  The
dataset consists of a primary classification, along with a measure of
confidence up to 100\%, and a secondary classification.  The secondary
cover type is given as the most likely alternative to the primary type
\citep{Friedl2010}, but for purposes of our analysis we are taking a
more probabilistic view and incorporating all available information
from the base data.  Because we are aggregating the data up to
5-arcmin resolution there is no expectation that the sub-pixel
fractions at full resolution are spaitally specific, but in the
aggregate our characterization of each grid cell's composition will be
nuanced by this additional information.  The primary class covers at
least roughly 50-60\% of a given pixel $x$, and this percent is almost
certainly a monotonically increasing function of the confidence
measure $c$.  \todo{cite email from Friedl}.  For the purposes of this
analysis we assume that this dependence is linear. Thus, for the
primary and secondary cover types in a pixel:

$$
A_p(x) = A_{min} + (1 - A_{min}) c(x)
$$
$$
A_s(x) = 1 - A_p(x)
$$

where $0.50 \le A_{min} \le 0.60$ is primarily chosen based on an
interpretation of $c$.  Given that there are only a handful of
examples of $c < 0.20$ \todo{consider including histograms showing
  confidence distribution}, setting $A_{min} = 0.50$ is appropriate.
Certainly for a classification to be considered the primary it must
represent a bare majority of the area covered by that pixel at
minimum, and the distributions of confidences indicate that the vast
majority of pixels contain greater than 60\% of their area in the
primary under the rubric described above.  The equations are simplified
as follows by assuming this value for $A_{min}$.

$$
A_p(x) = \dfrac{1 + c}{2}
$$
$$
A_s(x) = 1 - A_p(x) = \dfrac{1-c}{2}
$$

% will this help?



Applying these formulae results in a map for each cover type where the
pixel values are the sub-pixel areas on the interval $[0,1]$.  The map
of the fraction of the primary cover type is visually equivalent to
that of the classification confidence level because the former is
simply a linear scaling and offset of the latter.  \autoref{fig:thumb_fracs} shows the result of calculating $A_p + A_s$ for each individual class.  

%\subsubsection{Detail Area}
%\label{sec:agg-detail-area}


\begin{figure}[hpt] 
\begin{center}
  

\includegraphics{fig_thumb_fracs}
\end{center} 
\caption{Sub-pixel fractions at original resolution for $A_{min}=0.5$}
\label{fig:thumb_fracs}
\end{figure} 

By way of comparison we also consider the trivial case of settimg
$A_{min} = 1$ which indicates that the secondary cover is ignored
altogether and the primary cover is taken to represent 100\% of the
pixel area.  \autoref{fig:thumb1_fracs} shows these difference.  The
effect of adjusting $A_{min}$ is subtle; we will examine it more
closely after aggregating to the 5$'$ grid.

\begin{figure}[hpt] 
\begin{center}
  

\includegraphics{fig_thumb1_fracs}
\end{center} 
\caption{Sub-pixel fractions at original resolution for $A_{min}=1$}
\label{fig:thumb1_fracs}
\end{figure} 

Computationally the process of converting the reclassified maps to
sub-pixel fractions at the desired 5-arcmin resolution is a three-step
process.  First we calculate the fraction of the primary cover type as
a function of the classification confidence as described above.  Next,
a sub-pixel fraction for each cover type is calculated at full
resolution, recognizing that the primary and secondary classes may be
identical after the reclassification, such as cases where the original
data indicated two different type of forests.  Aggregating to a
coarser resolution is a simple matter of calculating the mean of these
values over the intersecting pixels at the original resoution.
Because the desired 5$'$ resolution is a multiple of the original
15$''$ resolution the pixels are perfectly nested, which is
convenient for properly computing this mean.


\begin{figure}[hpt] 
\begin{center}
  

\includegraphics{fig_thumb_agg}
\end{center} 
\caption{Aggregated sub-pixel fractions for $A_{min}=0.5$}
\label{fig:thumb_agg}
\end{figure} 

\begin{figure}[hpt] 
\begin{center}
  

\includegraphics{fig_thumb1_agg}
\end{center} 
\caption{Aggregated sub-pixel fractions for $A_{min}=1$}
\label{fig:thumb1_agg}
\end{figure} 


Before proceeding further it is interesting to inspect the differences
between the aggregated maps for the chosen values of $A_{min}$ as
shown in \autoref{fig:thumb_agg_diff}.  Positive values indicate that
$A_{min} = 0.5$ resulted in a greater fraction.  The main message from
this chart is that considering the secondary cover class results in
greater mixture between the crop and mosaic classes because cropland
is reduced in the north of the detail area where it was dominant in
the primary land cover type, and simlarly for mosaic in the south.
The relative suitability of these choices for $A_{min}$ is discussed
in \autoref{cha:analysis}.



\autoref{fig:thumb_agg_diff} emphasizes the difference between the
choice of $A_{min}=0.5$ and $A_{min}=1.0$ for the calculation of the
sub-pixel fractions and their aggregation to 5$'$ with a difference
map.  Positive values in the map indicate areas where $A_{min}=0.5$
produced a greater value.  We see more clearly from this set of maps
that the effect of considering the secondary class results in a shift
of up to 10\% of total cell area from crop to mosaic in the north of
our detail area and vice versa for the southern portion.  This
decrease in the relative dominance of the primary class is expected as
we saw from the earlier maps (\autoref{fig:thumb_pri_facet} and
\autoref{fig:thumb_sec_reclass}) of the MLCT data which classes were
indicated by the secondary classes in those areas.

\begin{figure}[hpt]
\begin{center}
  
%def

\includegraphics{fig_thumb_agg_diff}
\end{center} 
\caption{ Difference of aggregated sub-pixel fractions}
\label{fig:thumb_agg_diff}
\end{figure} 

%\subsubsection{Analysis Area}
%\label{sec:agg-analysis-area}


We apply the same functions for calculating the 15$''$-resolution map
of the primary cover class as a function of the confidence level $c$
for the entire cUSA study area, converting those to per-class
fractions at the same extent and scale, and aggregating those values
to the 5$'$ grid.  The corresponding figures are not shown because the
decrease in relative resolution makes interpretation difficult.  Based
on the behavior that these functions exhibited over the detail area we
can be confident that they will perform correctly over the greater
extent.



\subsection{Mosaic decomposition}
\label{sec:decomposition}

The MLCT classification includes a type that is problematic for the
economic models for which this data set is intended, the ``cropland /
natural vegetation mosaic'' class.  This class is defined as a hybrid
of cropland and some mixture of natural covers (forest, shrub, or
open) with no single component exceeding 60\% \citep{Friedl2002} and
croplands generally comprising 40--60\% of pixel area \todo{cite Friedl
  email}. Being a hybrid of developed land use and natural land cover
we wish to differentiate the cropland from the natural vegetation in
order to calculate a more meaningful total for cropland area and
thereby eliminate the mosaic class from the final tabulation.  In the
present implementation of the reclassification and aggregation process
we are making three very simple assumptions about the composition of
area delineated as mosaic lands:

\begin{enumerate}
\item Mosaic land is 50\% cropland.
\item The other 50\% is a blend of forest, open, and shrub in
  proportion to the expression of those classes in the same 5-minute
  cell.
\item In the absence of such information we simply assume that the
  natural component of the mosaic is an equal blend of all three.
\end{enumerate}

The intention here is to make simplifying assumptions that will allow
us to proceed with the evaluation of this analytical framework.
Although it may be interesting to vary the proportion used to
calculate the proportion of mosaic land to be allocated to crop land
we have no principled basis for this as of yet, considering that the
defintion implies that this proportion is variable across the MLCT
rather than being some unknown single-valued quantity.  The choice of
the 50\% level reflects the assertion that the mosaic is a cultural
class grouped with cropland and urban in the IGBP classification
scheme without overstating the degree of development.  MLCT provides
adequate variability in this dimension by commonly pairing cropland
and mosaic in the primary/secondary class data.  The second assumption
imposes that 15$''$ mosaic cells' non-crop portion will have the same
relative composition of forest, open, and shrub as the as the
non-mosaic portion of the 5$'$ grid cell in which it falls. Therefore
mosaic pixels in a 5$'$ cell where only forest is found of the three
non-crop mosaic components will be allocated 50\% crop and 50\%
forest.  \autoref{fig:thumb_nomos} and \autoref{fig:thumb1_nomos} show
the effect of decomposing the mosaic class in this fashion for
$A_{min}$ values of 0.5 and 1.0 respectively.



\begin{figure}[hpt]
\begin{center}
  


\includegraphics{fig_thumb_nomos}
\end{center} 
\caption{Aggregated cover fractions after mosaic decomposition, $A_{min}=0.5$}
\label{fig:thumb_nomos}
\end{figure} 

\begin{figure}[hpt]
\begin{center}
  


\includegraphics{fig_thumb1_nomos}
\end{center} 
\caption{Aggregated cover fractions after mosaic decomposition, $A_{min}=1.0$}
\label{fig:thumb1_nomos}
\end{figure} 



\begin{figure}[hpt]
\begin{center}
  

\includegraphics{fig_thumb_nomos_diff}
\end{center} 
\caption{Differences of sub-pixel fractions after mosaic
  decomposition, positive when $f(A_{min} = 0.5)$ is greater}
\label{fig:thumb_nomos_diff}
\end{figure} 





Our hypothesis from the outset is that there is information worth
capturing in the secondary class and classification confidence level
provided by MLCT.  We will test this hypothesis in
\autoref{cha:analysis} but in order to do so we need an ``observed
truth'' to provide an independent standard by which to make a
comparison on the basis of overall reduction in error at the 5$'$ grid
cell level.  The following section describes such a data set which
will be held up against these MLCT-derived data sets in the next
chapter.

% more appropriately addressed in following chapter

% Both 5-arcmin data sets derived from the MLCT in this fashion
% overestimate cropland area relative to that indicated by Agland2000,
% but the $A_{min} = 0.5$ variant better portrays the spatial variation
% judging from a simple root-mean-squared-error (RMSE)
% test. \todo{illustrate/demonstrate the RMSE test on the 5-arcmin MLCT
%   data sets}

\section{Agricultural Lands in the Year 2000 (Agland2000)}
\label{sec:agland2000}


The data set described by \citet{Ramankutty2008}, referred to in this
paper as ``Agland2000'', is the product of an effort to merge
satellite-derived LULC classifications with census data of
agricultural actviity compiled at national or sub-national levels
according to availability on around the turn of the last century.  It
uses both an older version of the MLCT (known as BU-MODIS) and the
GLC2000 data set mentioned in \autoref{sec:background} and a mask
based on climatic criteria and delineations of protected areas to
allocate the census data to the 5$'$ grid for both cropland and
pasture.  The ``open'' class in this data set has been renamed from
``pasture'' in its creator's nomenclature, but it is clear from its
ditribution shown in \autoref{fig:agland} that it represents a
phenomena that is not apparent in the MLCT data, so we do not attempt
to use it or reconcile it here, rather only carry it along to a small
degree for sake of comparison.  We attribute this discrepancy to
commingling of managed pasture lands and natural open land in the MLCT
classification.  It is important to note that Agland2000 is used as an
input into the classification algorithm of the version of MLCT that we
are using here and acknowledge the possibility of circularity when
comparing the two, but because of its basis in census data we will use
the cropland component of Agland2000 as an ``observed truth'' for the
purposes of evaluating our incremental adjustments to the maps we
derive from MLCT in \autoref{cha:analysis}.
  
%def 

\begin{figure}[hpt]
\begin{center}
  

\includegraphics{fig_thumb_agland}
\end{center} 
\caption{Agland2000 distribution in detail area}
\label{fig:thumb_agland} 
\end{figure} 

\begin{figure}[hpt]
\begin{center}
  

\includegraphics{fig_agland_trim}
\end{center} 
\caption{Agland2000 distribution in cUSA study area}
\label{fig:agland} 
\end{figure} 


\begin{comment}
\section{Major Land Uses (MLU)}
\label{sec:mlu}

This is a tabular data set published by the Economic Research Service
(ERS) at the USDA of land acreages by various uses and covers at a
state level.  We hope to compare our results to this data on a
state-by-state basis in order as a check and possibly incorporate some
of its information as a refinement.

\end{comment}

\section{National Land-cover Database 2001 (NLCD)}
\label{sec:nlcd}

\citet{Homer2004}


The NLCD gives a higer-resolution (30m) snapshot of LULC circa 2001.
\todo{check whether/how urban, water, wetland are informed with priors
  in NLCD} Reclassifying and aggregating this data to 5-arcmin
resolution in a fashion similar to that used for the MLCT is expected
to give better estimations of aggregate area for detailed features
like rural transportation networks and small stream and wetland
features.  This will compensate for MLCT's bias against these finely
detailed structures due to it's resolution.  It is the availability of
this information that makes it difficult to apply this analysis beyond
the United States without access to a comparable data set with global
extents.  The analysis is restricted to the conterminous US because of
the relative paucity of agricultural activity in Hawaii and Alaska.

As with the MLCT the process of reclassification and aggregation is
performed for both the detail region and the complete region.

One limitation of the \texttt{raster} library for \texttt{R} that we
are using is that the aggregation function requires that the output
resolution be a multiple of the output resolution.  The 30m resolution
of the NLCD equates to 1.25361$''$ and so does not satisfy this
requirement.  This deficiency was addressed by resampling the input to
1.25$''$ resolution prior to export from \texttt{GRASS} for this
analysis using a nearest-neighbor sampling algorithm, which gives an
even factor of 240 between the two resolutions.

\todo{Incorporate Joshua's suggestion to show further NLCD detail to better illustrate the discrepancy in developed areas}


\subsection{Reclassification}
\label{sec:nlcd-reclass}

\missingfigure{NLCD reclassification table}


\begin{figure}[hpt] 
\begin{center}


\includegraphics{fig_thumb_nlcd_reclass}
\end{center} 
\caption{NLCD reclassified} 
\label{fig:thumb_nlcd_reclass} 
\end{figure} 

\begin{figure}[hpt] 
\begin{center}
  

\includegraphics{fig_thumb_nlcd_facet}
\end{center} 
\caption{NLCD covers shown separately, detail} 
\label{fig:thumb_nlcd_facet} 
\end{figure} 

\subsection{Aggregation}
\label{sec:nlcd-aggr}

The same code used for refactoring the MLCT when considering only the
primary cover type can be applied here.

Repeating this process for the entire study area is computationally
expensive due to the NLCD's high resolution.


 

\begin{figure}[hpt] 
\begin{center}
  


\includegraphics{fig_thumb_nlcd_agg}
\end{center} 
\caption{NLCD aggregated cover fractions, detail area}
\label{fig:thumb_nlcd_agg}
\end{figure} 




\begin{figure}[hpt] 
\begin{center}
  


\includegraphics{fig_nlcd}
\end{center} 
\caption{NLCD aggregated cover fractions}
\label{fig:nlcd}
\end{figure} 

\begin{comment}
\section{Cropland Data Layer (CDL)}
\label{sec:cdl}

\missingfigure{Table or chart showing CDL covereage for various years}

The CDL is only available for a small number of states in 2001.  If
time allows it might be good to compare what is available with our
results as another independent evaluation against a higher-resolution
data set.

\subsection{Reclassification}
\label{sec:cdl-reclass}



%  gdalbuildvrt -tr 0.0002777778 0.0002777778 -te -124.8333 24.5 -66.91667 49.33333 cdl_2001.vrt $(find . -name "*2001*")


\todo{Calculate CDL mask for 5-arcmin cells completely filled}
\todo(Calculate CDL aggregated in GRASS}




\missingfigure{CDL reclassification table}

\subsection{Aggregation}
\label{sec:cdl-aggr}
\end{comment}

\section{Harvested Area and Yields of 175 Crops (175crops2000)}
\label{sec:175crops2000}

\citet{Monfreda2008}

\missingfigure{Table of crops and types reproduced from \citep{Monfreda2008}}

\missingfigure{Summary table of crop aggregations for our model}

\todo{Address issue of smaller land mask for 175crops2000 and Agland2000}

This data set will provide the information needed to disaggregate the
cropland area taken from Agland2000.  It is not possible to use this
data directly because it reflects only harvested area and so ignores
various types of ancillary agricultural land, rather it will provide
proportions for the disaggregation at the grid cell level.  Rather
than considering the full array of 175 crops we will consider only
corn, soy, wheat, rice, and sugarcane individually, combine other
cereals into their own class, and combine all remaining crops as a
catch-all ``other'' category.  Field crops will be distinguished from
orchard / plantation crops that would likely fall under areas
classified by MLCT as forest or shrub in this step.


\begin{figure}[hpt] 
\begin{center} 


\includegraphics{fig_crops}
\end{center} 
\caption{175Crops2000 category maps} 
\label{fig:crops} 
\end{figure} 


%%% Local Variables: 
%%% mode: latex
%%% TeX-master: "thesis"
%%% End: 

% -*- mode: noweb; noweb-default-code-mode: R-mode; -*-








\graphicspath{ {analysis/} }

\chapter{Analysis}
\label{cha:analysis}

% , eval=TRUE>>=

\begin{Schunk}
\begin{Sinput}
> options( prompt= " ", continue= " ", width= 60)
 options(error= function(){
   recover()
   options( prompt= "> ", continue= "+ ", width= 80)
 })
 source( "~/thesis/code/analysis.R")
 source( "~/thesis/code/peel.R")
 texWd <- "~/thesis/analysis"
 rasterWd <- "~/thesis/data/analysis"
 dataPath <- "~/thesis/data"
 setwd( rasterWd)
 overwriteRasters <- FALSE
                                         # studyArea used to work out RMSE
                                         # calcs and tables
 ##studyArea <- "thumb"
 studyArea <- "mlct"
                                         # bands are numbered from one but
                                         # classes from zero.  Used for stacks/brick
                                         # where bands correspond to classes
 peelBands <- peelClasses +1
                                         # mask and agland exported from GRASS
                                         # no need to mask or crop
 cusaMask <- raster( "mask_cusa.tif")
 cusaExtent <- extent( cusaMask)
 thumbExtent <- extent( -( 83 +30 /60), -( 82 +25 /60),
                           42 +55 /60,     44  +5 /60 )
                                         # default raster() output
                                         # has geographic proj, full extent
                                         # by default
 world <- raster()
 res(world) <- 5/60
 grid <- raster( cusaMask)
 grid[] <- cellsFromExtent( world, grid)
 grid <- mask( grid, cusaMask)
 if( studyArea == "thumb") {
   cusaMask <- crop( cusaMask, thumbExtent)
 }
 acresFile <- paste( "acres",
                    paste( studyArea, ".tif", sep=""),
                    sep="_")
 if( overwriteRasters) {
   acres <- area( cusaMask) *247.105381
   acres <- writeRaster( acres,
                        filename= acresFile,
                        overwrite= TRUE)
 } else acres <- raster( acresFile)
 agland <- stack( list.files( paste( dataPath, "agland", sep="/"),
                             patt= "(cropland|pasture).tif$",
                             full.names= TRUE))
 layerNames(agland) <- c("crop", "open")
 agland <- setMinMax( agland)
 if( studyArea == "thumb") {
   agland <- crop( agland, thumbExtent)
 }
 agg05 <-
   brick( list.files( dataPath,
                     patt= paste( studyArea, "_Amin_0.5_agg.tif", sep=""),
                     full.names= TRUE))
 layerNames( agg05) <- names( peelClasses)
 nomos05 <-
   brick( list.files( dataPath,
                     patt= paste( studyArea, "_Amin_0.5_nomosaic.tif", sep=""),
                     full.names= TRUE))
 layerNames( nomos05) <- c( names( peelClasses)[ -8], "total")
 agg1 <-
   brick( list.files( dataPath,
                     patt= paste( studyArea, "1_agg.tif", sep=""),
                     full.names= TRUE)) 
 layerNames( agg1) <- names( peelClasses)
 nomos1 <-
   brick( list.files( dataPath,
                     patt= paste( studyArea, "1_Amin_1_nomosaic.tif", sep=""),
                     full.names= TRUE))
 layerNames( nomos1) <- c( names( peelClasses)[ -8], "total")
 ## nlcd <- stack( paste( paste( dataPath, "nlcd", "nlcd", sep="/"), names( peelClasses[ -8]), "5min.tif", sep="_"))
 
 nlcd <- stack( sapply( names( peelClasses[ -8]),
                       function( cover) {
                         list.files( paste( dataPath, "nlcd", sep="/"),
                                    patt= paste( "nlcd", cover, "5min.tif$", sep="_"),
                                    full.names= TRUE)
                       }))
 nlcd <- setMinMax( nlcd)
 nlcd <- crop( nlcd, cusaMask,
              ##filename= paste( getwd(), "nlcd.tif", sep="/"),
              filename= "nlcd.tif",
              overwrite= TRUE)
 layerNames(nlcd) <- names( peelClasses[ -8])
 rasterNames <- c( "agland", "nlcd", "agg05", "agg1", "nomos05", "nomos1")
 dataSets <- sapply( rasterNames, function( n) eval( parse( text=n)))
 areas <- llply( dataSets,
 function( d) {
   res <- cellStats( d *acres, sum)
   names( res) <- layerNames( d)
   res
 })
 ## llply( areas, function( a) melt( a, value.name= deparse( substitute( a))))
 
 areasDf <- ldply( areas, function( a) melt( t( as.data.frame( a))))
 ## covers in columns
 ## areasCt <- cast( areasDf, .id ~ X2, subset= X2 != "total", sum, margins="grand_col")
 ## rownames( areasCt) <- areasCt[, ".id"]
 ## areasCt <- areasCt[, -1]
 ## areasCt[, c( names( peelClasses), "(all)")]
 
 ## covers in rows
 areasCt <- cast( areasDf, X2 ~ .id, subset= X2 != "total", sum, margins="grand_row")
 rownames( areasCt) <- areasCt[, "X2"]
 areasCt <- areasCt[, -1]
 areasCt <- areasCt[ c( names( peelClasses), "(all)"), rasterNames]
 
           
 
       
       
\end{Sinput}
\end{Schunk}


\begin{Schunk}
\begin{Sinput}
 local({
   colnames( areasCt) <- c( "Agland2000", "NLCD",
                           "\\pbox[c][][c]{3in}{Aggregated\\\\$A_{min}=0.5$}",
                           "\\pbox[c][][c]{3in}{Aggregated\\\\$A_{min}=1.0$}",
                           "\\pbox[c][][c]{3in}{No Mosaic\\\\$A_{min}=0.5$}",
                           "\\smallskip\\pbox[c][][c]{3in}{No Mosaic\\\\$A_{min}=1.0$}")
   print( xtable( areasCt / 10^6, 
                 caption= "Total Acreages by Map and Cover", 
                 label= "tab:areas",
                 digits= 1),
         add.to.row= list( 
           pos= list( 0, nrow( areasCt)),
           command= rep("\\noalign{\\smallskip}", times= 2)),
         size= "small",
         sanitize.colnames.function= function(x) x)
 })
\end{Sinput}
% latex table generated in R 2.12.2 by xtable 1.5-6 package
% Wed Mar 16 16:55:43 2011
\begin{table}[ht]
\begin{center}
{\small
\begin{tabular}{rrrrrrr}
  \hline
 & Agland2000 & NLCD & \pbox[c][][c]{3in}{Aggregated\\$A_{min}=0.5$} & \pbox[c][][c]{3in}{Aggregated\\$A_{min}=1.0$} & \pbox[c][][c]{3in}{No Mosaic\\$A_{min}=0.5$} & \smallskip\pbox[c][][c]{3in}{No Mosaic\\$A_{min}=1.0$} \\ 
  \noalign{\smallskip} \hline
water & 0.0 & 96.5 & 74.3 & 75.0 & 74.3 & 75.0 \\ 
  forest & 0.0 & 513.2 & 344.7 & 353.6 & 410.8 & 429.9 \\ 
  shrub & 0.0 & 420.1 & 358.7 & 341.8 & 387.2 & 368.0 \\ 
  open & 557.1 & 429.6 & 516.9 & 545.8 & 538.7 & 561.9 \\ 
  wetland & 0.0 & 95.0 & 26.0 & 11.0 & 26.0 & 11.0 \\ 
  crop & 446.5 & 310.8 & 378.9 & 369.6 & 495.4 & 488.1 \\ 
  urban & 0.0 & 102.8 & 27.3 & 29.8 & 27.3 & 29.8 \\ 
  mosaic & 0.0 & 0.0 & 232.9 & 237.0 & 0.0 & 0.0 \\ 
  barren & 0.0 & 24.5 & 32.8 & 28.9 & 32.8 & 28.9 \\ 
  (all) & 1003.7 & 1992.5 & 1992.5 & 1992.5 & 1992.5 & 1992.5 \\ 
   \noalign{\smallskip} \hline
\end{tabular}
}
\caption{Total Acreages by Map and Cover}
\label{tab:areas}
\end{center}
\end{table}\begin{Sinput}
 
 
 
\end{Sinput}
\end{Schunk}


After decomposing the mosaic class The MLCT indicates
495.4Ma (200.5Mha) of cropland for
$A_{min}=0.5$ and 488.1Ma (197.5Mha) for
$A_{min}=1.0$ in the cUSA in 2001. 

Pasture indicated by Aglands2000 appears to be a broader
classification than that of the NLCD's pasture class because much of
the grazing land east of the Mississippi river counted in the
Aglands2000 pasture map is absent in the NLCD pasture class.

Aglands2000 indicates roughly
Ma (Mha) of cropland.  The
inability of the MLCT data set to resolve rural transportation
networks, minor settlements, and small water or wetland features is a
major contribution to the surplus of cropland acreage indicated by the
MLCT.  Due to its greater resolution, ~30m vs. ~500m, the NLCD is
better suited at discerning developed areas in rural landscapes
ranging from rural roads to farmsteads to small communities that do
not show up in the MLCT data. There is a total area of roughly 74 Ma
(30 Mha) of development remaining after subtracting the MLCT urban
class from all developed classes in the NLCD where the NLCD shows
greater development after they have both been aggregated to the
5-arcmin grid. Applying this area as an offset to the cropland area in
Aglands2000 brings us closer to the expected acreage under cultivation
in 2001, although this assumes that all of that development intersects
with MLCT cropland area.


The purpose for processing the MLCT for two values of $A_{min}$ as
described in the previous chapter is to evaluate whether or not
information from the secondary cover type contributes positively to
the accuracy of the data set we seek to synthesize.  The primary
objective of this synthesis is to achieve accuracy in cropland
distribution.  Because the cropland layer in the Agland2000 data set
is derived from county-level production census statistics we adopt
this as the ground truth and will endeavor to adjust our product
accordingly.  Although MLCT overstates cropland acreage for both
$A_{min}=0.5$ and $A_min=1.0$ the discrimination among the two is made
by the distribution of errors rather than the aggregate error.

\missingfigure{error map for ``nomos'' vs. Agland2000 crop}

These maps show the cell-by-cell differences between the MLCT-derived
data set that we have calculated after mosaic decomposition and the
Agland2000 cropland map.  TO summarize and compare these errors we
calculate the root of the mean squared error (RMSE) given by:

$$
\operatorname{RMSE}=\sqrt{\frac{\sum_{i=1}^{n}(\hat\theta_i-\theta_i )^2}{n}}
$$

where $\hat\theta_i$ are the predictions derived from the respective
MLCT derivations and $\theta_i$ are the observations taken from the
Agland2000 data set.


\begin{Schunk}
\begin{Sinput}
 rmseDf <- ldply( list("nomos05", "nomos1"),
                 function( brickName) {
                   rmseRast( getPeelBand( get( brickName), "crop"),
                            unstack( agland)[[1]])
                 })
 rmseDf <- cbind( c( 0.5, 1.0), rmseDf)
 colnames( rmseDf) <- c( "$A_{min}$", "RMSE")
 
 
\end{Sinput}
\end{Schunk}



\begin{Schunk}
\begin{Sinput}
 print( xtable( rmseDf,
               caption= "RMSE, MLCT vs. Agland2000 crop",
               label= "tab:rmse",
               digits= c( 0, 1, 3)),
       include.rownames= FALSE,
       sanitize.colnames.function= function(x) x)
\end{Sinput}
% latex table generated in R 2.12.2 by xtable 1.5-6 package
% Wed Mar 16 16:55:45 2011
\begin{table}[ht]
\begin{center}
\begin{tabular}{rr}
  \hline
$A_{min}$ & RMSE \\ 
  \hline
0.5 & 0.165 \\ 
  1.0 & 0.180 \\ 
   \hline
\end{tabular}
\caption{RMSE, MLCT vs. Agland2000 crop}
\label{tab:rmse}
\end{center}
\end{table}\begin{Sinput}
 
\end{Sinput}
\end{Schunk}

The results on Table \ref{tab:rmse} indicate that $A_{min}=0.5$ is
more representative of the distribution of cropland because although
the total area indicated is higher there is less error on a
cell-by-cell basis indicating that it does a better job of
representing the spatial distribution than $A_{min}=1.0$.  Later when
we recalculate the cell proportions by accepting the values for
cropland area from Agland2000 as truth we can expect minimal
distortion in reconciling its landscape with that given by MLCT.  From
this point forward we will consider only the statistics derived from
setting $A_{min}=0.5$ for the aggregation of the MLCT data due to this
improved fit with Agland2000 cropland and its full consideration of
all information imparted by the MLCT data.


\section{NLCD Offsets}
\label{nlcd_offsets}


From Table \ref{tab:areas} it is apparent that the MLCT results are
negatively biased in the areas assigned to water, wetland, and urban
features relative to the NLCD.  It is clear from visual inspection
that features of these classes tend to have smaller characteristic
dimensions which causes them to be overlooked in the the MLCT
classification.  The most obvious example are the rural transportation
networks in areas delineated by the Public Land Survey System (PLSS)
where roads have been laid out on a regular grid of square miles
called sections.  In the PEEL classification this infrastructure is
included in the urban class as another form of developed land.

The process of merging this information from the NLCD is as follows:

\begin{enumerate}
  \item Create a mask comprised of pixels classified as water, wetland,
    or urban in the reclassified NLCD
  \item Resample the MLCT layers to NLCD resolution (\~1.25 arcsecs)
    using this mask
  \item Compute class-by-class offsets by accepting each NLCD pixel in
    the mask as a positive increment and each in the MLCT as a negative
    in proportion to the shares given by the formulas for $A_{pri}$ and
    $A_{sec}$.  Pixels outside the mask or where the data sets agree are
    assigned a zero value in this step.
  \item Aggregate these offsets to 5-arcmin resolution by taking the
    mean of offset values across a given output grid cell
  \item Add these offsets to the aggregated MLCT maps prior to the
    mosaic decomposition step
  \item Recalculate the mosaic decomposition
\end{enumerate}

\begin{Schunk}
\begin{Sinput}
 ## rebuild the thumb object from the previous chapter
 
 ## thumb <- mlctList( "thumb_2001_lct1.tif", 
 ##                    "thumb_2001_lct1_sec.tif", 
 ##                    "thumb_2001_lct1_pct.tif")
 
 thumbRasters <- list.files( dataPath, "thumb_2001.*_(reclass|pct)",
                            recursive=TRUE, full.names=TRUE)[ c(2,3,1)]
 names( thumbRasters) <- names(formals( mlctList))
 thumb <- do.call( mlctList, as.list( thumbRasters))
 thumb$Amin <- 0.5
 thumb$Ap <- raster( list.files( dataPath, "thumb_Amin_0.5.tif", full.names=TRUE))
 thumb$agg <- brick( list.files( dataPath, "thumb_Amin_0.5_agg.tif", full.names=TRUE))
 thumbNlcd <- list( pri=
                   raster( list.files( dataPath, "thumbNlcd_reclass.tif",
                                      recursive=TRUE, full.names=TRUE)))
\end{Sinput}
\end{Schunk}



\begin{Schunk}
\begin{Sinput}
 thumbNlcdMask <-
   if( overwriteRasters) {
     calc( thumbNlcd$pri,
          function( x) {
            ifelse( x %in% peelClasses[ c( "water", "wetland", "urban")],
                   x, NA)
          },
          datatype= "INT2U",
          overwrite= TRUE,
          filename= "thumbNlcdMask.tif",
          progress= "text")
   } else {
     raster( "thumbNlcdMask.tif")
   }
 thumbNlcdMask <- setMinMax( thumbNlcdMask)
 
\end{Sinput}
\end{Schunk}

\missingfigure{ NLCD mask facet map}

\begin{Schunk}
\begin{Sinput}
 thumbResamp <-
   if( overwriteRasters) {
     resample( stack( thumb$pri, thumb$sec),
              stack(thumbNlcd$pri, thumbNlcd$pri),
              method= "ngb",
              datatype= "INT2U",
              overwrite= TRUE,
              filename= "thumbResamp.tif",
              progress= "text")
   } else raster( "thumbResamp.tif")
 thumbResamp <- setMinMax( thumbResamp)
 thumbResampAp <-
   if( overwriteRasters) {
     resample( thumb$Ap, thumbNlcd$pri,
              method="ngb",
              datatype= "FLT4S",
              overwrite= TRUE,
              filename= "thumbResampAp.tif",
              progress= "text")
   } else raster( "thumbResampAp.tif")
 thumbNlcdMlct <-
   if( overwriteRasters) {
     mask( stack( thumbResamp, thumbResampAp),
          thumbNlcdMask,
          #datatype= "INT2U",
          overwrite= TRUE,
          filename= "thumbNlcdMlct.tif",
          progress= "text")
   } else raster( "thumbNlcdMlct.tif")
 thumbNlcdMlct <- setMinMax( thumbNlcdMlct)
\end{Sinput}
\end{Schunk}

\missingfigure{ MLCT resampled and masked facet(?) map(s?)}

\begin{Schunk}
\begin{Sinput}
 ##crosstab( thumbNlcdMlct, thumbNlcdMask)
 
 
 thumbOffsetsInputAp <-
   if( overwriteRasters) {
     brick( stack( thumbNlcdMlct,
                  thumbNlcdMask),
           filename= "thumbOffsetsInputAp.tif",
           overwrite= TRUE,
           progress= "text")
   } else brick( "thumbOffsetsInputAp.tif")
 ##offsetCalcFunWater <- offsetCalcFun(0)
 
 
 # The next few assignments end with "Ap" to signify that
 # the variable A_p is considered, thereby incorporating
 # information from both primary and secondary covers.
 
 # Prior implementation only considered primary cover in
 # calculating these offsets
 
 offsetCalcFunAp <- function( class) {
   fun <- function( st) {
     pri <- st[ 1]
     sec <- st[ 2]
     Ap <- st[ 3]
     nlcd <- st[ 4]
     result <- matrix( 0, nrow= 1, ncol= 9)
     if( !is.na( pri) &&nlcd ==class) {
       result[ 1, pri +1] <- -Ap
       result[ 1, sec +1] <- Ap -1
       result[ 1, nlcd +1] <- result[ 1, nlcd +1] +1
     }
     result
   }
   fun
 }
 thumbOffsetsAp <-
   sapply( grep( "water|wetland|urban",
                names(peelClasses), value=TRUE),
          function( cover) {
            fn <- paste( "thumbOffsetsAp",
                   paste( cover, "tif", sep= "."),
                   sep= "_")
            print( paste(cover, fn))
            if( overwriteRasters || !( file.access( fn) ==0)) {
              calc( thumbOffsetsInputAp,
                   fun= offsetCalcFunAp( peelClasses[[ cover]]),
                   datatype= "FLT4S",
                   overwrite= TRUE,
                   filename= fn,
                   progress= "text")
            } else brick( list.files( patt=fn, full.names= TRUE))
          })
 thumbOffsetsAp <-
   sapply( names( thumbOffsetsAp),
          function( cover) {
            fn <- paste( "thumbOffsetsAp",
                        cover, "agg.tif", sep= "_")
            print( paste( cover, fn))
            if( overwriteRasters || !( file.access( fn) ==0))
              aggregate( thumbOffsetsAp[[ cover]],
                        fact= 5/60 /res( thumbOffsetsAp[[ cover]]),
                        expand= FALSE,
                        filename= fn,
                        datatype= "FLT4S",
                        overwrite= TRUE,
                        progress= "text")
            else brick( list.files( patt=fn, full.names= TRUE))
          })
 thumbOffsetsAp <-
   sapply( thumbOffsetsAp,
          function( r) {
            layerNames( r) <- names( peelClasses)
            r
          })
 ## thumbOffsetsApTotal <-
 ##   writeRaster( thumbOffsetsAp$water +
 ##               thumbOffsetsAp$wetland +
 ##               thumbOffsetsAp$urban,
 ##               filename= "thumbOffsetsAp_total.tif",
 ##               overwrite=TRUE)
 
 
 thumbOffsetsApTotal <-
   do.call( overlay,
           c( unlist( thumbOffsetsAp, use.names= FALSE),
                fun= sum,
                filename= "thumbOffsetsAp_total.tif",
                overwrite= TRUE,
                progress= "text"))
 thumbAdj <- thumb
 thumbAdj$agg <- 
   overlay( thumbAdj$agg, thumbOffsetsApTotal,
           fun= sum,
           filename= "thumbAdj.tif",
           overwrite= TRUE)
 thumbAdj  <- decomposeMosaic( thumbAdj, overwrite= overwriteRasters, progress= "text")
 
\end{Sinput}
\end{Schunk}

\missingfigure{Facet map of thumb offsets}

\missingfigure{Difference map, thumb adjusted vs. original after mosaic decomposition}


Due to performance constraints it was not possible to carry out this
operation on the full cUSA study area.  The equivalent operation of
resampling the MLCT to the NLCD resolution of 1.25 arcsecs,
calculating the offsets for water, wetland, and urban (developed)
features implemented in a Bash script for use in the GRASS GIS
environment is given in the appendix.

\todo{Reference / hyperlink NLCD offset GRASS script in appendix}


The resulting offsets are added to the aggregated fractions calculated
from the MLCT with $A_{min}=0.5$.  


\begin{Schunk}
\begin{Sinput}
 offsetFile <- path.expand( paste( rasterWd, "nlcd_offset.tif", sep="/"))
 offset <- brick( offsetFile)
 offset <- setMinMax( offset)
 layerNames(offset) <- names( peelClasses)
 ## mlctAdj <- list( Amin=0.5)
 ## mlctAdj$agg <- 
 ##   overlay( agg05, offset,
 ##           fun= sum,
 ##           filename= "agg05Adj.tif",
 ##           overwrite= TRUE)
 
 mySum <- function( ...) {
   res <- sum( ...)
   res[ res > 1] <- 1
   res[ res < 0] <- 0
   res
 }
                                         # use this function to clean up any over/underruns
                                         # resulting from floating point math
 
 mlctAdj <- list( Amin=0.5)
 mlctAdj$agg <- 
   overlay( agg05, offset,
           fun= mySum,
           filename= "agg05Adj_mySum.tif",
           overwrite= TRUE)
 layerNames( mlctAdj$agg) <- names( peelClasses)
 mlctAdj  <- decomposeMosaic( mlctAdj, overwrite= overwriteRasters, progress= "text")
 
\end{Sinput}
\end{Schunk}

\missingfigure{Facet map of cUSA NLCD offsets}




\begin{Schunk}
\begin{Sinput}
 # reuse area table code from above; better to implement a function?
 
 rasterNames2 <- c( "agland", "nlcd", "agg05", "nomos05",
                   "offset", "mlctAdj$agg", "mlctAdj$nomos")
 dataSets2 <- sapply( rasterNames2,
   function( n) eval( parse( text=n)))
 areas2 <- llply( dataSets2,
   function( d) {
     res <- cellStats( d *acres, sum)
     names( res) <- layerNames( d)
     res
   })
 areasDf2 <- ldply( areas2, function( a) melt( t( as.data.frame( a))))
 ## covers in rows
 areasCt2 <- cast( areasDf2, X2 ~ .id, subset= X2 != "total", sum, margins="grand_row")
 rownames( areasCt2) <- areasCt2[, "X2"]
 areasCt2 <- areasCt2[, -1]
 areasCt2 <- areasCt2[ c( names( peelClasses), "(all)"), rasterNames2]
 
\end{Sinput}
\end{Schunk}

Following these algebraic acrobatics it seems prudent to check our
accounting with some simple arithmetic.  Working backwards from the
final result of adding NLCD-derived offsets to the raster stack
derived from the MLCT with $A_{min}=0.5$ and decomposing the remaining
mosaic fractions into their constituent cover types, subtracting the
deltas that came from the mosaic decomposition, subtracting offsets
calculated from the NLCD, and subtracting the aggregated MLCT data
from the previous chapter 
\todo{hyperlink to section where MLCT was aggregated} 
should produce zeroes everywhere, plus or minus the noise of floating
point math.


\begin{Schunk}
\begin{Sinput}
 ## check that everything balances
 ## output of decomposeMosaic is not brick()ed properly
 ## in the sense that the layer set is incomplete
 ## and out of order
   
 zeroes <- cusaMask
 zeroes[] <- 0
 restack <- function( peelBrick) {
   u <- unstack( peelBrick)
   names( u) <- layerNames( peelBrick)
   r <- do.call( stack,
           llply( names( peelClasses),
                 function( cover) {
                   if( is.null( u[[ cover]]))
                     zeroes
                   else
                     u[[ cover]]
                 }))
   layerNames( r) <- names( peelClasses)
   r
 }
                                         # restack() takes any of the bricks/stacks from
                                         # previous functions and rearranges the layers
                                         # to match the PEEL classes, inserting layers of
                                         # zeroes as needed
 
 
 ## check <- llply( mlctAdj[ c("nomos", "delta", "agg")], restack)
 ## names(check) <- NULL
 ## do.call( overlay, c(check , fun=function( n, d, a) n-d-a))
 
 restackOverlay <- function( rasterList, fun) {
   l <- llply( rasterList, restack)
   names( l) <- NULL
   do.call( overlay, c( l, fun=fun))
 }
                                         # restackOverlay() runs its arguments through restack()
                                         # and applies a function to its outputs
 
 ## restackOverlay( mlctAdj[ c("nomos", "delta", "agg")],
 ##                function( n, d, a) n-d-a)
 
 ## restackOverlay( list( mlctAdj$agg, offset, agg05),
 ##                function( a2, o, a) a2-o-a)
 
\end{Sinput}
\end{Schunk}


\begin{Schunk}
\begin{Sinput}
 check <- restackOverlay( c( mlctAdj[ c("nomos", "delta")], offset, agg05),
                function( n, d, o, a) n-d-o-a)
 layerNames(check) <- names( peelClasses)
 check
\end{Sinput}
\begin{Soutput}
class       : RasterBrick 
dimensions  : 298, 695, 9  (nrow, ncol, nlayers)
resolution  : 0.08333333, 0.08333333  (x, y)
extent      : -124.8333, -66.91667, 24.5, 49.33333  (xmin, xmax, ymin, ymax)
projection  : +proj=longlat +ellps=WGS84 +datum=WGS84 +no_defs +towgs84=0,0,0 
values      : in memory
min values  : -2.1e-10 -1.1e-16 -1.1e-16 -1.1e-16 -3.9e-09 -1.1e-16 -1.1e-16 -1.1e-16 -1.1e-16 
max values  : 3.6e-11 1.9e-09 2.3e-09 2.5e-09 5.4e-11 5.1e-10 1.1e-16 1.8e-09 1.4e-09 
\end{Soutput}
\begin{Sinput}
 
\end{Sinput}
\end{Schunk}

\begin{Schunk}
\begin{Sinput}
 checkTable <-
   xtable( cbind( class=peelClasses,
                 min=minValue( check),
                 max=maxValue(check)),
          caption= "Balance of adjustment fractions and original MLCT aggregation", 
          label= "tab:restack_check")
 digits( checkTable) <- c( 0, 0,-2,-2)
 print( checkTable)
\end{Sinput}
% latex table generated in R 2.12.2 by xtable 1.5-6 package
% Wed Mar 16 16:57:17 2011
\begin{table}[ht]
\begin{center}
\begin{tabular}{rrrr}
  \hline
 & class & min & max \\ 
  \hline
water & 0 & -2.09E-10 & 3.58E-11 \\ 
  forest & 1 & -1.11E-16 & 1.92E-09 \\ 
  shrub & 2 & -1.11E-16 & 2.32E-09 \\ 
  open & 3 & -1.11E-16 & 2.49E-09 \\ 
  wetland & 4 & -3.88E-09 & 5.36E-11 \\ 
  crop & 5 & -1.11E-16 & 5.05E-10 \\ 
  urban & 6 & -1.11E-16 & 1.11E-16 \\ 
  mosaic & 7 & -1.11E-16 & 1.85E-09 \\ 
  barren & 8 & -1.11E-16 & 1.35E-09 \\ 
   \hline
\end{tabular}
\caption{Balance of adjustment fractions and original MLCT aggregation}
\label{tab:restack_check}
\end{center}
\end{table}\begin{Sinput}
   
\end{Sinput}
\end{Schunk}

To assess whether the process of adding in the NLCD offsets has
improved overall cropland accuracy we can perform the same error
calculation from above and extend Table~\ref{tab:rmse} with the new
result, giving us Table~\ref{tab:rmse2}.

\begin{Schunk}
\begin{Sinput}
                                         # add the RMSE for the new crop map
                                         # and an indication of the NLCD offsets' presence
   
 rmseDf <-
   cbind( offset=c( TRUE, FALSE, FALSE),
         rbind( c( 0.5,
                  rmseRast( getPeelBand( mlctAdj$nomos, "crop"),
                           unstack( agland)[[ 1]])),
               rmseDf))
                                         # add the RMSE for the open class
 rmseDf <-
   cbind( rmseDf,
         rmseOpen=ldply( list(mlctAdj$nomos, nomos05, nomos1),
                 function( brickVar) {
                   rmseRast( getPeelBand( brickVar, "open"),
                            unstack( agland)[[ 2]])
                 }))
 colnames(rmseDf)[ c(3,4)] <- c( "$RMSE_{crop}$", "$RMSE_{open}$")
 print( xtable( rmseDf,
               caption= "RMSE, MLCT vs. Agland2000 crop with NLCD offsets",
               label= "tab:rmse2",
               digits= c( 0, 0, 1, 3, 3)),
       include.rownames= FALSE,
       sanitize.colnames.function= function(x) x)
\end{Sinput}
% latex table generated in R 2.12.2 by xtable 1.5-6 package
% Wed Mar 16 16:57:22 2011
\begin{table}[ht]
\begin{center}
\begin{tabular}{lrrr}
  \hline
offset & $A_{min}$ & $RMSE_{crop}$ & $RMSE_{open}$ \\ 
  \hline
TRUE & 0.5 & 0.150 & 0.235 \\ 
  FALSE & 0.5 & 0.165 & 0.242 \\ 
  FALSE & 1.0 & 0.180 & 0.267 \\ 
   \hline
\end{tabular}
\caption{RMSE, MLCT vs. Agland2000 crop with NLCD offsets}
\label{tab:rmse2}
\end{center}
\end{table}\begin{Sinput}
 
\end{Sinput}
\end{Schunk}

\todo[caption=Should the RMSE tables be rearranged?]{Would it make
  more sense to have the row order and independent variables (first
  three) reversed in Table \ref{tab:rmse} and \ref{tab:rmse2}?}

Seeing that this modifcation to the data set has improved our overall
accuracy of the distribution of croplands the next step is to examine
the total areas for all classes compared with the input data sets.  


\begin{Schunk}
\begin{Sinput}
 local({
   colnames( areasCt2) <- c( "Agland2000", "NLCD", "MLCT", 
                            "\\pbox[c][][c]{3in}{MLCT\\\\No Mosaic}",
                            "\\pbox[c][][c]{3in}{NLCD\\\\Offsets}", 
                            "\\pbox[c][][c]{3in}{MLCT\\\\Adjusted}",
                            "\\pbox[c][][c]{3in}{\\smallskip{}MLCT\\\\Adjusted\\\\No Mosaic}")
   print( xtable( areasCt2 / 10^6, 
                 caption= "Effect of NLCD offsets on total acreages, $A_{min}=0.5$",
                 label= "tab:areas2",
                 digits= 1),
         size= "small",
         add.to.row= list( 
           pos= list( 0, nrow( areasCt)),
           command= rep("\\noalign{\\smallskip}", times= 2)),        
         sanitize.colnames.function= function(x) x)
         ##,
         ##        sanitize.text.function= function(x) x))
   ##,
   ##      floating= FALSE)
 })
\end{Sinput}
% latex table generated in R 2.12.2 by xtable 1.5-6 package
% Wed Mar 16 16:57:22 2011
\begin{table}[ht]
\begin{center}
{\small
\begin{tabular}{rrrrrrrr}
  \hline
 & Agland2000 & NLCD & MLCT & \pbox[c][][c]{3in}{MLCT\\No Mosaic} & \pbox[c][][c]{3in}{NLCD\\Offsets} & \pbox[c][][c]{3in}{MLCT\\Adjusted} & \pbox[c][][c]{3in}{\smallskip{}MLCT\\Adjusted\\No Mosaic} \\ 
  \noalign{\smallskip} \hline
water & 0.0 & 96.5 & 74.3 & 74.3 & 23.7 & 98.0 & 98.0 \\ 
  forest & 0.0 & 513.2 & 344.7 & 410.8 & -52.4 & 292.3 & 346.4 \\ 
  shrub & 0.0 & 420.1 & 358.7 & 387.2 & -31.0 & 327.6 & 349.8 \\ 
  open & 557.1 & 429.6 & 516.9 & 538.7 & -20.6 & 496.3 & 515.9 \\ 
  wetland & 0.0 & 95.0 & 26.0 & 26.0 & 80.5 & 106.5 & 106.5 \\ 
  crop & 446.5 & 310.8 & 378.9 & 495.4 & -37.3 & 341.6 & 437.5 \\ 
  urban & 0.0 & 102.8 & 27.3 & 27.3 & 79.7 & 107.1 & 107.1 \\ 
  mosaic & 0.0 & 0.0 & 232.9 & 0.0 & -41.1 & 191.8 & 0.0 \\ 
  barren & 0.0 & 24.5 & 32.8 & 32.8 & -1.4 & 31.4 & 31.4 \\ 
  (all) & 1003.7 & 1992.5 & 1992.5 & 1992.5 & -0.0 & 1992.5 & 1992.5 \\ 
   \noalign{\smallskip} \hline
\end{tabular}
}
\caption{Effect of NLCD offsets on total acreages, $A_{min}=0.5$}
\label{tab:areas2}
\end{center}
\end{table}\begin{Sinput}
 
 
\end{Sinput}
\end{Schunk}



\begin{Schunk}
\begin{Sinput}
 ## thumbAgland <- crop( agland,
 ##                     extent(-83.5, -(82+25/60), 42+55/60, 44+5/60),
 ##                     filename= "thumbAgland.tif",
 ##                     progress="text")
 
 nomosCrop <- getPeelBand( mlctAdj$nomos, "crop")
 aglandCrop <- unstack( agland)[[ 1]]
 ## ones <- zeroes
 ## ones[] <- 1
 
 ## noncropFactor <- ( ones -aglandCrop) /( ones -nomosCrop)
 
 noncropFactor <-
   overlay( aglandCrop, nomosCrop, fun=
           function( a, n) {
             ## if( is.na( a) & !is.na( n)) 1
             ## else ( 1 -a) /( 1 -n)
             ifelse( is.na( a) & !is.na( n), 1,
                    ( 1 -a) /( 1 -n))
           })
 ## cropFactor <- aglandCrop /nomosCrop
 
 cropFactor <-
   overlay( aglandCrop, nomosCrop, fun=
           function( a, n) {
             ifelse( is.na( a) & n >0, 1,
                    ifelse( is.na( a) |
                           a >=0 & n ==0, 0, a /n))
           }  )
             ## ## if( is.na( a) & !is.na( n)) 1
             ## ## else if( a >0 & n ==0 ) 0
             ## ## else a /n
             ## ones <- is.na( a) & !is.na( n)
             ## factor <- ifelse( a >0 & n ==0, 0, a /n)
             ## factor[ ones] <- 1
             ## factor
 
 nomosClasses <- layerNames(mlctAdj$nomos)[-9]
                                         # leaves out 'total'
                                         # mosaic is already gone
 factorStack <- 
   stack( llply( nomosClasses =="crop",
                function( isCrop) {
                  if( isCrop)
                    cropFactor
                  else
                    noncropFactor
                }))
 cropOffset <-
   overlay( aglandCrop, nomosCrop, fun=
           function( a, n) {
             ifelse( is.na( a) & n >= 0, 0,
                    ifelse( a >0 & n ==0, a, 0))
           })
             ## if( is.na( a) & n > 0) n
             ## else if( a >0 & n ==0) a
             ## else 0
 
 
 offsetStack <-
   stack( llply( nomosClasses =="crop",
                function( isCrop) {
                  if( isCrop)
                    cropOffset
                  else
                    zeroes
                }))
 ## aglandComplete <- mlctAdj$nomos *factorStack +offsetStack
 
 aglandComplete <-
   if( overwriteRasters) {
     overlay( stack( unstack( mlctAdj$nomos)[ -9]),
             factorStack,
             offsetStack, 
             fun= function( x, m, b) m *x +b,
             filename= "aglandComplete.tif",
             overwrite= TRUE,
             progress= "text")
   } else brick( list.files( rasterWd,
                          patt="^aglandComplete.tif$",
                          full.names=TRUE))
 layerNames( aglandComplete) <- names(peelClasses)[-8]
 agcMap <- coverMaps( aglandComplete, 0.4) +
   coord_equal() +
   ## facet_wrap( ~variable, ncol=1)
   facet_grid( variable ~ .)
 
 
\end{Sinput}
\end{Schunk}

\begin{figure} 
\begin{center} 

\begin{Schunk}
\begin{Sinput}
 setwd( texWd)
 ggsave( "fig_agc.png", width=4.5, height=8)
 
 ## png( file="fig_agc.png",
 ##     height= 8, width= 4.5, units= "in"  # no effect
 ##     res= 300)
 ## print( agcMap)
 ## dev.off()
 
\end{Sinput}
\end{Schunk}

\includegraphics{fig_agc}
\end{center} 
\caption{Agland Complete cover maps} 
\label{fig:agc} 
\end{figure} 



blah blah blah

\begin{Schunk}
\begin{Sinput}
 setwd( rasterWd)  
 rmseDf <- 
   cbind( agland=c( TRUE, rep(FALSE, times=3)),
         rbind( c( TRUE, 0.5,
                  rmseRast( getPeelBand( aglandComplete, "crop"),
                           unstack( agland)[[ 1]]),
                  rmseRast( getPeelBand( aglandComplete, "open"),
                           unstack( agland)[[ 2]])),
               rmseDf))
 rmseDf <- within(rmseDf, offset <- as.logical( offset))
                                         # had to change offset column back
                                         # to true/false;  maybe this can be
                                         # avoided with list() instead of c()
 
 rmseXt <- xtable( rmseDf,
                  caption= "RMSE of Agland Complete vs. Agland2000",
                  label= "tab:rmse3",
                  digits= c( 0, 0, 0, 1, 3, 3))
                                         # looks like some kind of bug in xtable()
                                         # manual correction:
 rmseXt$agland <- rmseDf$agland
 rmseXt$offset <- rmseDf$offset
 print( rmseXt,
       include.rownames= FALSE,
       sanitize.colnames.function= function(x) x)
\end{Sinput}
% latex table generated in R 2.12.2 by xtable 1.5-6 package
% Wed Mar 16 16:57:49 2011
\begin{table}[ht]
\begin{center}
\begin{tabular}{llrrr}
  \hline
agland & offset & $A_{min}$ & $RMSE_{crop}$ & $RMSE_{open}$ \\ 
  \hline
TRUE & TRUE & 0.5 & 0.000 & 0.207 \\ 
  FALSE & TRUE & 0.5 & 0.150 & 0.235 \\ 
  FALSE & FALSE & 0.5 & 0.165 & 0.242 \\ 
  FALSE & FALSE & 1.0 & 0.180 & 0.267 \\ 
   \hline
\end{tabular}
\caption{RMSE of Agland Complete vs. Agland2000}
\label{tab:rmse3}
\end{center}
\end{table}\begin{Sinput}
 
\end{Sinput}
\end{Schunk}


\begin{Schunk}
\begin{Sinput}
 areasCt3 <- acreageTable( c( rasterNames2[ c( 1, 2, 4, 7)], "aglandComplete"))
 local({
   colnames( areasCt3) <- c( "Agland2000", "NLCD",
                            "\\pbox[c][][c]{3in}{MLCT\\\\No Mosaic}",
                            "\\pbox[c][][c]{3in}{\\smallskip{}MLCT\\\\Adjusted\\\\No Mosaic}",
                            "AgC")
   print( xtable( areasCt3 / 10^6, 
                 caption= "Agland Complete (AgC) acreages, $A_{min}=0.5$",
                 label= "tab:areas3",
                 digits= 1),
         size= "small",
         add.to.row= list( 
           pos= list( 0, nrow( areasCt)),
           command= rep("\\noalign{\\smallskip}", times= 2)),        
         sanitize.colnames.function= function(x) x)
   ##,
   ##      floating= FALSE)
 })
\end{Sinput}
% latex table generated in R 2.12.2 by xtable 1.5-6 package
% Wed Mar 16 16:57:55 2011
\begin{table}[ht]
\begin{center}
{\small
\begin{tabular}{rrrrrr}
  \hline
 & Agland2000 & NLCD & \pbox[c][][c]{3in}{MLCT\\No Mosaic} & \pbox[c][][c]{3in}{\smallskip{}MLCT\\Adjusted\\No Mosaic} & AgC \\ 
  \noalign{\smallskip} \hline
water & 0.0 & 96.5 & 74.3 & 98.0 & 98.7 \\ 
  forest & 0.0 & 513.2 & 410.8 & 346.4 & 365.6 \\ 
  shrub & 0.0 & 420.1 & 387.2 & 349.8 & 352.7 \\ 
  open & 557.1 & 429.6 & 538.7 & 515.9 & 472.9 \\ 
  wetland & 0.0 & 95.0 & 26.0 & 106.5 & 109.1 \\ 
  crop & 446.5 & 310.8 & 495.4 & 437.5 & 447.9 \\ 
  urban & 0.0 & 102.8 & 27.3 & 107.1 & 114.7 \\ 
  NA &  &  &  &  &  \\ 
  barren & 0.0 & 24.5 & 32.8 & 31.4 & 31.1 \\ 
  (all) & 1003.7 & 1992.5 & 1992.5 & 1992.5 & 1992.5 \\ 
   \noalign{\smallskip} \hline
\end{tabular}
}
\caption{Agland Complete (AgC) acreages, $A_{min}=0.5$}
\label{tab:areas3}
\end{center}
\end{table}\begin{Sinput}
 
 
\end{Sinput}
\end{Schunk}


\todo[inline,caption={Decide what to do with old code in Analysis
  chapter}]{The remaining code in this draft of the Analysis chapter
  has fallen out of date.  The maps are based on old data that has
  been replaced and the tables are probably not going to appear given
  that the code sections will not be evaluated.  The idea of finding
  correlations across a set of difference map still seems promising,
  so I am leaving it in the source code for now.}


To assess the impact of this step on the overall accuracy it is useful
compare the errors and biases of our newly derived ``Agland
Complete''(AgC) data set for all cover classes before and after the
adjustment of the cell-by-cell cropland areas to match Agland2000.





 

\begin{Schunk}
\begin{Sinput}
 ## calculate RMSE/bias summaries
 ## comparing everything to NLCD
 
 
 rmseAgc <- rmseSummary( function(c) paste(  "agc", c, sep="_"),
                        function(c) paste( "nlcd", c, sep="_"))
 rmseAs00 <- rmseSummary( function(c) paste( "mlct_2001", c, "As00", sep="_"),
                         function(c) paste( "nlcd", c, sep="_"))
 rmseAs05 <- rmseSummary( function(c) paste( "mlct_2001", c, "As05", sep="_"),
                         function(c) paste( "nlcd", c, sep="_"))
\end{Sinput}
\end{Schunk}

\begin{Schunk}
\begin{Sinput}
 ## t( rmseAgc)
 ## t( rmseAs00)
 ## t( rmseAs05)
 
 print( xtable( t( rmseAgc), 
               caption= "Errors and Biases of Aglands Complete relative to NLCD",
               label= "tab:ebagc",
               digits= c( 0, 2, -2, 0, 0)))
 print( xtable( t( rmseAs00), 
               caption= "Errors and Biases of MLCT, $A_s = 0.0$ relative to NLCD",
               label= "tab:ebmlct00",
               digits= c( 0, 2, -2, 0, 0)))
 print( xtable( t( rmseAs05), 
               caption= "Errors and Biases of MLCT, $A_s = 0.5$ relative to NLCD",
               label= "tab:ebmlct05",
               digits= c( 0, 2, -2, 0, 0)))
\end{Sinput}
\end{Schunk}

\begin{Schunk}
\begin{Sinput}
 ## agcAvgAcres <-
 ##   sapply( paste( "agc_", covers, sep=""),
 ##          function( map) {
 ##            mapRast <- raster( as.spgdf( handle( map)))
 ##            return( cellStats( areaAcres( mapRast), sum)
 ##                   /( ncell( mapRast) - cellStats( mapRast, 'countNA')))
 ##          })
 
 
 ## getting ready to plot
 
 stackAgc <- stackHandles( grepHandles(  "^agc"))
 attr( stackAgc, "layernames") <-  covers
 stackNlcd <- stackHandles( grepHandles( "^nlcd"))
 attr( stackNlcd, "layernames") <-  covers
 stackDiff <- stackAgc -stackNlcd
 attr( stackDiff, "layernames") <-  covers
 
\end{Sinput}
\end{Schunk}

\begin{figure} 
\begin{center} 

\begin{Schunk}
\begin{Sinput}
 ##spgdfNlcd <- as.spgdf( stackNlcd)
 ##names( spgdfNlcd) <- layerNames( stackNlcd)
 setwd( texWd)
 png( file="fig_nlcd.png")
 print( coverMaps( stackNlcd, 0.4))
 dev.off()
\end{Sinput}
\end{Schunk}

\includegraphics{fig_nlcd}
\end{center} 
\caption{NLCD cover maps} 
\label{fig:nlcd} 
\end{figure} 

\begin{figure} 
\begin{center} 

\begin{Schunk}
\begin{Sinput}
 ##spgdfDiff <- as.spgdf( stackDiff)
 ##names( spgdfDiff) <- layerNames( stackDiff)
 setwd( texWd)
 png( file="fig_diff.png")
 print( coverMaps( stackDiff, 0.4) + 
       scale_fill_gradientn( "diff", colours= rev( brewer.pal( 11, "BrBG")), 
                            limits= c( 0.1, -0.1),
                            breaks= seq( 0.1, -0.1, by= -0.02)))
 dev.off()
\end{Sinput}
\end{Schunk}

\includegraphics{fig_diff}
\end{center} 
\caption{Difference maps, Aglands Complete minus NLCD} 
\label{fig:diff} 
\end{figure} 

\begin{figure} 
\begin{center} 

\begin{Schunk}
\begin{Sinput}
 ## look for correlations across the difference maps
 
 corDiff <- cor( as.data.frame( as.spgdf( stackDiff))[,1:8])
 colnames( corDiff) <- unlist( lapply( 
                                      strsplit( colnames( corDiff), "\\."), 
                                      function( x) return( x[ 2])))
 rownames( corDiff) <- unlist( lapply( 
                                      strsplit( rownames( corDiff), "\\."), 
                                      function( x) return( x[ 2])))
 ord <- order.dendrogram( as.dendrogram( hclust( dist( corDiff))))
 corDiffPlot <- 
   ggplot( melt( corDiff),
          aes( x=X1, y=X2, fill= value)) +
   geom_tile() +
   theme_bw() +
   opts( panel.grid.minor= theme_blank(),
        panel.grid.major= theme_blank(),
        panel.background= theme_blank(),
        axis.title.x= theme_blank(),
        axis.text.x= theme_text( angle= 90, hjust=1),
        axis.title.y= theme_blank()) +
   scale_x_discrete( limits= colnames(corDiff)[ord]) +
   scale_y_discrete( limits= colnames(corDiff)[ord]) +
   scale_fill_gradientn( "cor", colours= rev( brewer.pal( 11, "BrBG")), 
                        limits= c( 1.0, -1.0),
                        breaks= seq( 1.0, -1.0, by= -0.2))
 setwd( texWd)
 png( file="fig_cordiff.png")
 print( corDiffPlot)
 dev.off()
\end{Sinput}
\end{Schunk}
\includegraphics{fig_cordiff}
\end{center} 
\caption{Correlations across cover type in difference maps} 
\label{fig:cordiff} 
\end{figure} 

The elements of the matrix have been reordered according to the
clustering forumla given in \citet[sec. 6.2.3]{Sarkar2008} in order to
achieve a degree of visual clustering among the correlation vectors.

\begin{Schunk}
\begin{Sinput}
 options( prompt= "> ", continue= "+ ", width= 80)
\end{Sinput}
\end{Schunk}

%%% Local Variables: 
%%% mode: latex
%%% TeX-master: "thesis"
%%% End: 



\chapter{Conclusions}
\label{cha:conclusions}



\backmatter

\bibliography{thesis}

\appendix

\chapter{Source Code: Data Sets}

\section*{R helper functions}
\lstinputlisting{code/peel.R}

\section*{R code embedded in the chapter}
\lstinputlisting{datasets.R}

\chapter{Source Code: Analysis}

\section*{R helper functions}
\lstinputlisting{code/analysis.R}

\section*{R code embedded in the chapter}
\lstinputlisting{analysis2.R}

\end{document}