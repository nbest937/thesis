% -*- mode: noweb; noweb-default-code-mode: R-mode; -*-








\graphicspath{ {data/} }


\chapter{Data}
\label{cha:data}

This chapter presents some summary descriptions of the various data
sets that are relevant to this analysis and further discussion on how
they were manipulated.

The general approach with the classified land cover data sets (MLCT,
NLCD, CDL) is to reclassify their categories and aggregate the new
classification to the 5-arcmin grid.  The purpose of the
reclassification is to reduce the number of classes and have a uniform
set of classes across data sets.  The challenge in this is that
classification defintions are sometimes subtly different which makes
direct comparison across data sets difficult.  In this process we
convert classified maps whose pixels have discrete values to a stack
of maps, one map per class, whose pixels have real number values on
the interval $[0,1]$ and are constrained to sum to unity for each
pixel through the stack.  In the general case of the MLCT data product
the process converts two discrete, thematic variables and one
continuous variable, those being a primary covery type, a secondary
cover type, and classification confidence level respectively, into a
set of continuous variables representing fractional areas for the
cover types in the siplified classification system.  In other cases
the process is simplified by considering only a primary thematic layer
and performing the aggregation without a secondary cover type or
confidence level by which to relate them.

\missingfigure{Generate a summary table of data sets (raster/tabular, resolution, citation)}

\section{MODIS Land Cover Type (MLCT)}
\label{sec:mlct}

In preparation for this analysis we prepared the 2001 MLCT data by patching
together the tiles as delivered in the equal-area sinusoidal
projection, reprojecting that mosaic to geographic coordinates, and
extracting a subset for the conterminous United States (cUSA).  The
subset is defined as the set of 5-arcmin grid cells that intersect
with the cUSA polygon in the Global Administrative Areas (GADM) vector
data set, which includes the water bodies on the American side of the
international border across the Great Lakes, but does not extend to
oceanic waters beyond the coastal grid cells that intersect with the
land mass.  

The expectation is that the analytical approach developed here will be
applied globally in the future.

To illustrate the process of converting the MLCT data from its
original representation we are including maps of an area of
southeastern Michigan to show greater detail through each step of the
process.  We chose this region for its diversity of land covers and
uses, diveristy of agricultural commodities across its significant
cropland area, and its familiarity to our principal author, being his
brthplace.  In this section we will demonstrate the process of
converting the MLCT data from its native form, consisting of primary
cover type, classification confidence for the primary cover, and
secondary (alternate) cover type at 15-arcsec resolution, to a stack
of cover fractions at 5-arcmin resolution using the simplified
cover/use classification specified by the PEEL model. 

\subsection{Reclassification}
\label{sec:mlct-reclass}

The following table shows the mapping of the IGBP classes used in the
original MLCT data to the simplified classification designed for the
PEEL model.

\missingfigure{MLCT reclassification table}


